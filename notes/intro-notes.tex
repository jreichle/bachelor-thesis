%%%%%%%%%%%%%%%%%%
% methodology section

find 5 possible solutions and compare

Add: how was problem defined
only after (incl.) 2017

Searching:
search strategy for specific review: $\rightarrow$ provide reasoning behind choices
\begin{itemize}
    \item search tearms
    \item databases
    \item inclusion \& exclusion criteria: year of publification, type of article, journal, research
        quality
\end{itemize}
write all decisions down, how is search documented
The research is then extended to existing practical solutions for similar practical problems as
well as proposed architectures from academic research.
Documentation of search process and selection, asses quality of search process and selection

Analysis:
how is data prepared for analysis
analyse problem: find requirements
type of information that needs to be extracted: what works, why, not necesseraly how
literature review (state of the art)
how? top journals of last few years
Theoretical knowledge about the models as well as practical information about results for

Synthesis

after review:
discuss different approchoaches pros and cons for problem
methodogical limitations: practice is always different, \ldots


Introduction to Design Science~\citep{johannesson_introduction_2021}
Design Science (design and develop artifacts, contextual knowledge about artifacts) <-> empirical science (describe, explain, predict)

Define Artifact
- define problem to solve: current state --- destination  state
- Adress stakeholders?
- Functional Requirements
- Bin-functional requirements
- Artifact structure-> inner workings
- Take environment into account (side effects)

How to evaluate results?
- Find similar work
Which kind of contribution
- Improvement
- Routine Design
- Exaptation
- Invention

Knowledge types -> purpose
- Definitional knowledge: provides basic concepts required to express knowledge -> define concepts, without claims of existence
- Descriptive knowledge: describes, summarizes, generalizes, classifies observations-> describe, without claim of explanation or prediction
- Explanatory knowledge: answers how objects behave and why events occur- often in form of cause effect chains -> explain past, without claiming predictions
- Predictive knowledge: predict outcomes based on underlying factors without explaining causal relationship -> predict, not explain
- Explanatory and predictive knowledge
- Prescriptive knowledge: models and methods that help solve practical problems
Knowledge forms

Types of artifacts
- Constructs
- Models
- Methods
- Instantiations

Design theory: generate knowledge about produced artifact
- Purpose and scope
- Constructs
- Principle of form and function -> abstract blueprint or architecture that describes architecture
- Artifact mutability -> changes in state
- Testable propositions -> propositions about instantiation
- Justificatory knowledge -> knowledge that provides justification for design
- Principles of implementation
- Expository instantiation

Research strategy: Case Study
- Investigates multiple factors, events, relationships that occur in a real world case
- Characterization
    - Focus on one instance
    - Focus on depths
    - Natural setting
    - Relationships and processes
    - Multiple sources and methods
- Can be exploratory, descriptive, explanatory
Research strategy: action research
- Address practical problems
- Characteristics
    - Focus on practice
    - Change in practice
    - Active practitioner participation
    - Cyclical process: diagnosis, planning, intervention, reflection
    - Action outcomes and research outcomes
- Challenge: generalize results, remaining impartial
Research strategy	: Simulation
Research strategy	: mathematical and logical proof

Data collection
- Quantitative (numeric) - qualitative (text, …)
- Mixed approach -> use another strategy to verify results of first
- Strategies
    - Observation: directly observe phenomena, problem -> subjectivity -> do observation schedule (S. 165)
    - Documents: academic stuff

Data analysis
- Quantitative (numbers) - qualitative (text, …)
- Quantitative data: categorical, ordinal, interval, ratio

Deduce: specific conclusion from general principle
Induce: general principle from specific observations


Result: which knowledge type which knowledge form
Read research paper for clues for introduction
