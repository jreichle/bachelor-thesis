For comparison: which Benchmark Dataset fits the problem the best?
\begin{itemize}
    \item~\cite{liao_mask_2020}:
        \begin{itemize}
            \item Rotated ICDAR 2013 (changed normal icdar): rotation robustness
            \item Total-Text: shape ropustness
            \item MSRA-TD500: aspect ratio ropustness
        \end{itemize}
    \item~\cite{yang_learning_2021}: commonly used for oriented text: ICDAR2015, ICDAR2017 MLT,
        MSRA-TD500
\end{itemize}

Scene text detection and recognition~\cite{long_scene_2021}
Detection:
\begin{itemize}
    \item sub-text components: \\
        better flexibility and generalization over shapes and aspect ratios\\
        drawback: module or post-processing step used to group segments into text instances
        may be vulnerable to noise and the efficiency
\end{itemize}
Recognition
\begin{itemize}
    \item CTC:\ less dependant on language models and has better character to pixel alignment
    \item Encoder-Decoder: decoder is an implicit language model: can incorporate more linguistic priors
    \item both: assume text is straight and can therefore not adapt to irregular text
        challenge: represent oriented characters and curved text that are distributed over a
        2-dimensional space
            (rather than 1-dim/horizontal) in order to fit decoding modules (whose decodes require
            1-dimensional inputs)
        \item evaluation of recognition methods falls behind the time robustness of recognition when
            cropped with slightly differend bounding box is seldom verified
\end{itemize}
Character level annotations are more accurate and better. However, most existing datasets do not
provide character- level annotating. Since characters are smaller and close to each other,
character-level annotation is more costly and inconvenient.
