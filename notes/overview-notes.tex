no transformers $\rightarrow$ self-attention mechanism is too computationally expensive???

`The great advances that have been made in fields such as computer vision and speech recognition,
have been accom- plished by replacing a modular processing pipeline with large neural networks
that are trained end-to-end [37]. In essence, transparency is traded for accuracy.
This is an unavoidable reality.'\citep{arpteg_software_2018}

Scene Text Detection and Recognition~\cite{long_scene_2021}
Deep learning $\rightarrow$ frees researchers from hand-crafting features, simplifies pipeline,
    improves performance

Difference: two-part pipeline vs end-to-end
\begin{itemize}
    \item two-part / Detection \& Recognition: detector passes cropped image to recognizer
    \item end-to-end: detector passes cropped feature maps to recognizer $\rightarrow$ end-to-end
            training
\end{itemize}

\textbf{Text detection and localization}
can be generalized under object detection --- generally: one-staged or two staged methods
scene text detection algorithms often inspired bz object detectors --- difference:
        text is homogeneous as a whole and characterized bz its locality
stages of development
\begin{itemize}
    \item early DL approaches: long and slow pipelines, design methodology is bottom up
    \item methods inspired by object detection:
        \begin{itemize}
            \item modifying region proposal and bounding box regression modules to localize text
                directliy
            \item consist of stacked convolutional layers that encode image into feature maps
            \item spatial location at the feature maps corresponds to region of input image
            \item feature maps are fed into classifier for prediction of existance and localization
        \end{itemize}
        see p.5 / p.165 for specific papers
    \item Methods based on Sub-text components
        \begin{itemize}
            \item any part of a text insance is still text
                $\rightarrow$ only predict sub-text components and then assemble into a text instance
            \item use NN to predict local attributes or segments $\rightarrow$ postprocessing for
                re-construction
            \item in comparison to early DL-approaches: rely more on NN and shorter pipelines
            \item different approaches
                \begin{itemize}
                    \item pixel-level: learn dense prediction map --- does pixel belong to
                        any text instances
                    \item component-level: predict local region of text instance (overlapping one or
                        more characters)
                    \item character-level: learn segmentation map for character centers and links
                        between them, centers and links predicted in form of gaussian heat map,
                        problem: requires iterative weak supervision, but real-world datasets rarely
                        equipped with character-level labels
                \end{itemize}
                sub-text components: \\
                better flexibility and generalization over shapes and aspect ratios\\
                drawback: module or post-processing step used to group segments into text instances
                may be vulnerable to noise and the efficiency
        \end{itemize}
\end{itemize}

\textbf{Recognition --- Text transcribing and converting into linguistic symbols}
use CNNs to encode images into feature space
different approaches in text content decoding module:
challenge: represent oriented characters and curved text that are distributed over a 2-dimensional space
    (rather than 1-dim/horizontal) in order to fit decoding modules (whose decodes require
    1-dimensional inputs)
\begin{itemize}
    \item Connectionist Temporal Classification (CTC)
        \begin{itemize}
            \item input images are viewed as sequence of vertical pixel frames
            \item network outputs per-frame prediciton $\rightarrow$ probability distribution of
                label types for each frame
            \item CTC-rule is applied to edit per-frame prediction to a text string
            \item training end-to-end: sum of negative log propability of all possible per-frame
                predictions that generate target sequence by CTC-rules
            \item less dependant on language models and has better character to pixel alignment
        \end{itemize}
    \item Encoder-Decoder Framework
        \begin{itemize}
            \item encoder RNN reads input sequence and passes final latent state to decoder RNN
                which generates output in auto-regressive way
            \item often combined with attention mechanism
            \item decoder is an implicit language model: can incorporate more linguistic priors
        \end{itemize}
    \item adaptions for irregular text recognition
        \begin{itemize}
            \item Spatial Transformer Networks (predict text bounding polyglons with fc-layers for
                thin-plate-spline transformations to rectify irregular input into more canonical form)
                $\rightarrow$ Sequence Recognition Network
            \item \ldots
        \end{itemize}
    \item other methods
        \begin{itemize}
            \item perform word recognition by classifying image into pre-defined set of vocabulary
            \item improve occlusion cases: transformer-based semantic reasoning module
        \end{itemize}
\end{itemize}
evaluation of recognition methods falls behind the time robustness of recognition when cropped
with slightly differend bounding box is seldom verified

\textbf{End-to-end system}
also known as text spotting systems, profiting from idea of designing differentiable computation graphs
\begin{itemize}
    \item Two-step pipeline:
        \begin{itemize}
            \item recent work goes away from character level and towards word or line level
            \item first generate proposal using text detection model, thenrecognize using
                text recognition model
            \item detected words are cropped from the image $\rightarrow$ detection and recognition
                are two separate steps
        \end{itemize}
    \item Two-stage pipeline
        \begin{itemize}
            \item more recent
            \item cropp feature maps instead of images and feed to recognition modules
        \end{itemize}
    \item One-Stage Pipeline:
        \begin{itemize}
            \item predict character, text bounding boxes and character type segmentation
                maps in parallel
            \item text bb are used to group character boxes to form final word transcription results
        \end{itemize}
\end{itemize}

What is wrong with scene text recognition~\cite{baek_what_2019}
Vertical texts: most of current STR models assumes hori-
zontal text images, and thus structurally could not deal with
vertical texts. Some STR models [28, 5] exploit vertical in-
formation also, however, vertical texts are not clearly cov-
ered yet. Further research would be needed to cover vertical
texts.


Generative Pretraining from Pixels~\citep{chen_generative_2021}
\begin{itemize}
    \item unsupervised representation learning (approach transfered from NLP)
    \item training of sequence Transformer to auto-regressively predict pixels without incorporating
        knowledge of 2D input structure
    \item Active part: GPT-2 scale model learns image representations and performs extremely well even
        when compared to supervised models
\end{itemize}

Learning High-Precision Bounding Box for Rotated Object Detection via Kullback-Leibler
Divergence~\citep{yang_learning_2021}
\begin{itemize}
    \item Deductive approach to rotated object detection
    \item box is `translated' to 2D-Gaussian $\rightarrow$ KLD with prediction and true gaussian as Loss
    \item LIMIT:\ cannot be directly applied to quadrilateral detection
\end{itemize}

which aspects to compare? quantitative, qualitative
