\chapter{Introduction}\label{ch:intro}
\section{Motivation}
Digitization is the transformation of analog information into a digital
representation~\citep{imgrund_approaching_2018}.
Information systems in conjunction with digitization help optimize efficiency and productivity
for business performance, as well as reduce costs~\citep{imgrund_approaching_2018}.
This facilitates the growing need for automation~\citep{imgrund_approaching_2018}.
Additionally, a comprehensive information system with such digitized information allows a company
to aggregate and share information to harness its benefits~\citep{goodhue_impact_1992}.
However, before reaping the rewards, the information must first be digitized.
Take technicians, for example, who work in the field with different equipment.
It is helpful to digitize the labels of such equipment, to keep an overview of the
inventory~\citep{abramowicz_business_2019}.
The automated process of digitizing such data is called \ac{OCR}, the concept of extracting typed,
handwritten, or printed text from an image~\citep{zhao_improving_2020}.
Techniques for this concept have improved a lot due to the advances in
\ac{DL}~\citep{zhao_improving_2020}.
Compared to traditional methods, \ac{DL} improves automation, effectiveness, and
generalization~\citep{chen_text_2021}.
Applying these new capabilities and finding the right solution in the space of \ac{DL} for the
use case of extracting information of labels is the focus of this thesis.
This is an exciting task as the performance of \ac{OCR} systems in natural scenes is still
challenging~\citep{zhao_improving_2020, chen_text_2021}.
Such scenes entail natural scenes captured by a camera~\citep{chen_text_2021, baek_what_2019}.
Factors such as complex backgrounds, noise, perspective, and variability in fonts, colors, and sizes,
of scene texts, complicate the process~\citep{hu_gtc_2020,chen_text_2021,baek_what_2019}.
In these conditions, \ac{OCR} is known as \ac{STS}~\citep{long_scene_2021}.

\section{Problem Description}\label{se:problem}
This thesis aims to create an overview of possible \ac{DL} techniques that facilitate
finding a solution for digitization.
Which state-of-the-art \ac{DL} approaches for \ac{STS} are viable for the use case of extracting
textual label data from real-world images.

It is difficult to assess how well a \ac{DL} approach performs before being implemented and tested
on the specific problem or representative dataset~\citep{arpteg_software_2018}.
This justifies creating an overview rather than pointing out a single approach that is
deemed the most promising.
There are different categories of approaches~\citep{chen_text_2021,long_scene_2021}.
The categories must be distinguished and explained.
How the respective issues for the steps are solved must be
identified from the literature, listed, and explained alongside.

The definition of the viability of an approach must be determined to judge a technique's shortcomings.
What qualities such as detecting alpha-numeric strings or suitability despite
inadequate image conditions must a solution have~\citep{ghosh_visual_2017, hu_gtc_2020}?

The approaches' shortcomings must be identified to compare them.
This analysis cannot be performed with quantitative data, as results from different studies are
not safely comparable~\citep{baek_what_2019,arpteg_software_2018,long_scene_2021}.
Therefore, the comparison must leverage qualitative data concerning the different approach
categories.

The article by~\cite{ashmore_assuring_2021} defines four phases of the \ac{ML} lifecycle:
Data Management, Model Learning, Model Verification, and Model Deployment.
Only the substage Model Selection from Model Learning will only be looked at in the scope of this
thesis.
The substage aims at selecting a suitable \ac{ML} model to perform a given
task~\citep{ashmore_assuring_2021}.
The substages loss function selection, training, and hyperparameter
selection~\citep{ashmore_assuring_2021} will not be examined.
Other aspects such as data analysis, implementation, training, deployment, and maintenance of a
solution in a production environment shall not be performed either.
The selection of approaches is only concerned with their potential performance and not
necessarily what steps must be executed to reach said performance.
Note that is this work is concerned with end-to-end recognition, which aims to detect and recognize
every text instance, as opposed to word spotting, which aims to find known
words only~\citep{chen_text_2021,karatzas_icdar_2015}.

The overview and subsequent analysis create a foundation for finding the right solution.
However, it does not contain any claims about the degree of goodness or the certainty of
solving the given problem.
Based on this thesis, further verification, implementation, and testing can then be performed.

\section{Methodology}\label{se:methodology}
The methodology of this thesis can be labeled as a literature review~\citep{snyder_literature_2019,
torraco_writing_2005}.
The goal is to provide an overview of current \ac{DL} techniques that can help choose
which to implement and test to solve the specific problem of \ac{STS}, defined in
Section~\ref{se:problem} and further developed in Chapter~\ref{ch:problem}.

The research question guiding the process is most crucial~\citep{snyder_literature_2019}:
What \ac{DL} approaches for scene text \ac{OCR} are there, and what are their shortcomings concerning
the problem of extracting textual label data from images taken in real-world conditions?

It is important to report how the information was found and
synthesized~\citep{torraco_writing_2005}.
Therefore, each section in the overview contains a paragraph about said information.

Before getting into the current research level, a foundation of knowledge about the field must be
laid.
A taxonomy is created to facilitate the clarity of the subsequent provision of current research and
the analysis.
The taxonomy is helpful to classify and give context to innovations in the field.
It is formed with the help of overview literature and other related research in the field.

The research strategy is most important for a literature
review~\citep{snyder_literature_2019}.
The strategy for current innovations includes determining databases and keywords
used and exclusion criteria that are enforced~\citep{torraco_writing_2005}.
Innovations are explored through searching in the Google Scholar database.
Additionally, literature is selected through citations for and literature that has already been
identified as important.
All research after 2018 which pertains to extracting scene text is regarded as relevant.
Standard \ac{OCR} solutions may not hold validity in practice, as the image and text conditions can
vary in the defined problem~\citep{chen_text_2021}.
A criterion for further examination is appropriate citations for the piece of literature
in question.
Another necessary criterion is that the paper contributes to the \ac{ML} model.
This extends to the whole pipeline from extraction features to the final result of the model.
The identified literature is synthesized into an overview aligned according to the taxonomy.

To improve the validity of the subsequent analysis, the problem is dissected further.
This includes analyzing the specific use case and researching which qualities have been
identified as generally critical for scene text approaches.
The qualities are taken from the literature covering \ac{ML} in general to literature
covering \ac{OCR} under challenging scene text conditions.

In the analysis, possibly viable approaches are compared with the required qualities defined
in Chapter~\ref{ch:problem}.
The comparison is organized according to the taxonomy, which facilitates clarity and
comprehensibility.
The analysis thus shows which approaches are worthwhile to apply the whole \ac{ML} lifecycle to.

\section{Expected Results}
In addition to a deeper understanding of the problem and its detailed definition, the literature
review lays the foundation for finding the right approach for the extraction of textual
information from images with equipment labels through literature review.
The subsequent analysis highlights different approaches for their theoretical fit as a
solution.

In the following, the structure of this thesis is listed, and each chapter's expected
result is detailed along with its benefits for the overall objective of producing an overview of
state-of-the-art \ac{STS} relevant for the problem described in Section~\ref{se:problem}.
Chapter~\ref{ch:theoretical} lays the theoretical foundation for later chapters.
This includes general principles of \ac{DL} and, by extension \ac{ML}, and of \ac{STS}.
Chapter~\ref{ch:research} examines current research.
The overview is twofold: a taxonomy for the pipeline and its approaches as well as innovations in
current research.
The resulting overview can be viewed as a basis for a decision to implement a
practical solution.
In Chapter~\ref{ch:problem}, the problem from Section~\ref{se:problem} is addressed in more detail.
The result shall be a firm understanding of qualities that a solution must possess.
These requirements are the point of focus for the further examination of techniques and
enable the discussion in Chapter~\ref{ch:discussion}.
Here the results and the availability of a solution and the methodology of this work
are assessed critically.
The conclusion is a summary of the results compared to the expected results detailed in this chapter,
and an outlook for further research into the topic.
