\chapter{Introduction}\label{ch:intro}
\section{Motivation}
\ac{OCR} is the concept of extracting typed, handwritten or printed text
from an image.
Techniques for this concept have improved a lot due to the advances in the field of
\ac{DL}~\cite{zhao_improving_2020}.
When compared to traditional methods \ac{DL} improves automation, effectiveness and
generalization~\cite{chen_text_2021}.
\ac{DL} is a technology based on \acp{NN} where data is processed
in multiple layers to extract complex features to solve a given problem~\cite{shrestha_review_2019}.
\ac{DL} has only caught on in the recent years as the big computational cost has been met
by improvement in computer hardware as well as automatic feature
learning~\cite{ponti_everything_2017, chen_text_2021}.
Finding the right solution in the space of \ac{DL} and applying these new capabilities to
the use case of extracting information of labels is the focus of this thesis.
This is an interesting task as performance of \ac{OCR} systems in complex scenes is still
challenging~\cite{zhao_improving_2020}.
Such scenes entail natural scenes captured by a camera.
\ac{OCR} in these conditions is also known as \ac{STR}~\cite{chen_text_2021}.
Factors like complex backgrounds, noise, perspective and variability in fonts, colors and sizes,
of scene texts complicate the process~\cite{hu_gtc_2020,chen_text_2021}.
Therefore, it is critical to specify possible factors for the underlying problem and to find
criteria for evaluating feasibility.

\section{Problem description}\label{se:problem}
The basic problem of this thesis is finding a viable solution for the extraction of textual
information from images with equipment labels.
However, it is difficult to assess how well an approach performs before it has been implemented and
tested on the specific problem or dataset~\cite{arpteg_software_2018}.
Therefore, it is useful to propose several approaches that might solve the problem from
different angles and different properties.

The problem has to first be analyzed in depth in order to find viable approaches for the solution.
This includes defining requirements such as detecting alpha-numeric strings or suitability despite
suboptimal image conditions~\cite{ghosh_visual_2017, hu_gtc_2020}.
These requirements define properties that an approach must have in order to be classified as viable.
Thus the reasearch and subsequent discussion of techniques from end-to-end \ac{OCR} to dividing the
process into text detection and text recognition is centered around the requirements which are
given by the problem.

Subsequent aspects such as implementation, training, deployment and maintenance of a solution in a
production environment shall not be performed within the scope of this thesis.
However, because these aspects may vary depending on the approach, it is important to consider them when
discussing the viability for solving the problem.

% FIXME: add following
only model selection activity~\cite{ashmore_assuring_2021}
set into context to MLC lifecycle: Data Management, Model Learning (Model Selection, Training,
Transfer Learning, Hyperparameter Selection), Model Verificatoin

\section{Methodology}\label{se:methodology}
The methodology of this thesis can be described as a literature review.
As such, the research question guiding the process is most crucial: Which state of the art \ac{DL}
approaches for \ac{OCR} are viable for the use case of extracting textual label data from
images.
The following section describes how relevant literature is identified, analyzed and synthesised.

find 5 possible solutions and compare

Add: how was problem defined
only after (incl.) 2017

Searching:
search strategy for specific review: $\rightarrow$ provide reasoning behind choices
\begin{itemize}
    \item search tearms
    \item databases
    \item inclusion \& exclusion criteria: year of publification, type of article, journal, research
        quality
\end{itemize}
write all decisions down, how is search documented
The research is then extended to existing practical solutions for similar practical problems as
well as proposed architectures from academic research.
Documentation of search process and selection, asses quality of search process and selection

Analysis:
how is data prepared for analysis
analyse problem: find requirements
type of information that needs to be extracted: what works, why, not necesseraly how
literature review (state of the art)
how? top journals of last few years
Theoretical knowledge about the models as well as practical information about results for

Synthesis

after review:
discuss different approchoaches pros and cons for problem
methodogical limitations: practice is always different, \ldots


guiding values from studies (see problem)

\section{Expected results}
% FIXME: kürzen
% FIXME: weg von literature review -> problem with criteria, analysing and summarizing research,
%               find viable approaches and compare properties for 5
In addition to a deeper understanding of the problem and its detailed definition, the literature
review lays the foundation for finding the right approach for the extraction of textual
information from images with equipment labels through literature review.
In the subsequent analysis different approaches are highlight for their theoretical fit as a solution.

In the following section the structure of this thesis is listed and each chapter's expected
result is detailed along with its benefits for the overall objective of producing an overview of
state of the art \ac{OCR} relevant for the problem described in Section~\ref{se:problem}.
comprehension of the following chapters is gathered.
% FIXME: grammar above
This includes general principles of \ac{DL} and by extension \ac{ML} but also of \ac{OCR}.\@
In Chapter~\ref{ch:problem} the problem from Section~\ref{se:problem} is addressed in more detail.
The result shall be a firm understanding of functional and non-functional requirements both on the
technical and the business process side.
% FIXME: add exclusion criteria
These requirements are the point of focus for the further examination of \ac{OCR} techniques.
After laying the foundation, in Chapter~\ref{ch:research} current research in regards to the
identified requirements is examined.
The resulting overview can be viewed as a basis for a decision when it comes implementing a practical
solution.
Therefore it enables the discussion in Chapter~\ref{ch:discussion}.
Here not only the results and the availability of a solution but also the methodology of this work
are assessed critically.
The conclusion is a summary of the results compared to the expected results detailed in this chapter
as well as an outlook for further research into the topic.
