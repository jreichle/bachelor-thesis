\chapter{Introduction}\label{ch:intro}
\section{Motivation}
Optical Character Recognition is the concept of extracting typed, handwritten or printed text
from an image.
Techniques for this concept have improved a lot due to the advances in the field of deep
learning~\cite{zhao_improving_2020}.
Deep Learning is a technology based on Artificial Neural Networks where data is processed
in multiple layers to extract complex features and solve a given problem~\cite{shrestha_review_2019}.
Deep Learning has only caught on in the recent years as the big computational cost has been met
by the improvement in computer hardware~\cite{ponti_everything_2017}.
Finding the right solution in the space of deep learning and applying these new capabilities to
the use case of extracting information of equipment labels is the focus of this thesis.

\section{Problem description}\label{se:problem}
The central part of the bachelor thesis is finding the right approach for the extraction of textual
information of of images with equipment labels.
This includes defining functional requirements such as detecting rotated text but also non functional
requirements such as given computational power of mobile devices that are to use the solution.
These requirements define properties that a solution must have in order to be classified as viable.
Thus the discussion of techniques from end-to-end OCR to dividing the process into text detection and
text recognition is centered around the requirements which are given by the problem.
The research ranges from established solutions for similar problems to current research in the field.
% TODO: add more

Following aspects such as implementing, training, deploying and maintaining a solution in a
production environment shall not be subject of this thesis.

\section{Methodology}

The goal of this work is to implement and train a Deep Learning model to read in labels from photos.
The emerging artifact can be used to solve the problem detailed in chapter~\ref{se:problem}.
The expository instantiation is helpful to gain more understanding the artifact as it is common
in design science.
In particular this is justificatory knowledge on the design on the Deep Learning model and
Machine Learning way of approaching problems.
This is important in order to apply it and to optimize existing research to the specific problem.

The methodology is based on action research~\cite{johannesson_introduction_2021}.
It constists of a cycle of five phases: Diagnosis, Planning, Intervention, Evaluation, Reflection.
The first cycle will entail an exploratory data analysis which corresponds to the Diagnosis part.
Here it is important to recognize main characteristics of the images and to find outliers
and other potential problems~\cite{cox_translating_2017}.
The research is then extended to existing practical solutions for similar practical problems as
well as proposed architectures from academic research.
Theoretical knowledge about the models as well as practical information about results for
similar problems contribute to the discussion about which approach is the most promissing.
Combining architectures is also a viable possibility to solve the given problem.
This concludes the Planning phase and will lead to a model exaptation that evolves to be the
artifact at the center of this thesis.
The next step is implementing and training the chosen approach which.
Evaluation for of the current model follows.
Storing and analyzing results of training and cross validation as well as visualizing the training
progress is an important part of this.
In the Reflection stage it is decided whether a new cycle should be carried out.

From the second cycle on the first three phases change as there already is a model that is to be
improved.
This time the Diagnosis phase entails asking questions about the existing model: What worked?
Why did it work/not work? What needs to change?
Changes are planned and implemented accordingly.
The Evaluation and Reflection phases are not changing in the second cycle thus closing the loop.
The incremental adjustments to the model are made in order to improve the accuracy.
This includes possibly adjusting the architecture, hyperparameter tuning and preprocessing approaches
like image compression.

% TODO: add chapter oversight and their connections -> add next lines

\section{Expected results and outlook}
The research into the theoretical foundation of Deep Learning and into possible approaches leads
to a strong understanding of the underlying technology.
This is helpful to produce a comparison of approaches that is based on theoretical as well as
practical knowledge.
The goal is to find out which approach work best for the chosen practical problem and why that is the
case.
Implementation and training of the most promissing one is yielding the artifact this work revolves
around.
The process of optimization not only improves the solution to the problem (see~\ref{se:problem}) but
is also used to learn more about the implemented approach.
