\chapter{Introduction}\label{ch:intro}
\section{Motivation}
Digitization can be described as transforming analog information into a digital
respresentation~\citep{imgrund_approaching_2018}.
Information systems in conjunction with digitization help to optimize efficiency and productivity
for business performance, as well as to reduce costs~\citep{imgrund_approaching_2018}.
This facilitates the growing need for automation~\citep{imgrund_approaching_2018}.
Additionally, a comprehensive information system with such digitized information allows a company
to aggregate and share information to harness it~\citep{goodhue_impact_1992}.
However, before reaping the rewards, the information must be digitized in the first place.
Take technicians for example, who work in the field with different equipment.
It is useful to digitize the lables of such equipment, to keep an overview over the
inventory~\citep{abramowicz_business_2019}.
The automated process of digitizing such data is called \ac{OCR}, the concept of extracting typed,
handwritten or printed text from an image~\citep{zhao_improving_2020}.
Techniques for this concept have improved a lot due to the advances in the field of
\ac{DL}~\citep{zhao_improving_2020}.
When compared to traditional methods \ac{DL} improves automation, effectiveness and
generalization~\citep{chen_text_2021}.
Applying these new capabilities and finding the right solution in the space of \ac{DL} for the
use case of extracting information of labels is the focus of this thesis.
This is an interesting task as performance of \ac{OCR} systems in natural scenes is still
challenging~\citep{zhao_improving_2020, chen_text_2021}.
Such scenes entail natural scenes captured by a camera~\citep{chen_text_2021, baek_what_2019}.
Factors such as complex backgrounds, noise, perspective and variability in fonts, colors and sizes,
of scene texts complicate the process~\citep{hu_gtc_2020,chen_text_2021,baek_what_2019}.
In these conditions \ac{OCR} is known as \ac{STS}~\citep{long_scene_2021}.

\section{Problem Description}\label{se:problem}
The goal of this thesis is to create an overview over possible \ac{DL} techniques that facilitates
finding a solution for the process of digitization.
The research question guiding the process is most crucial:
Which state of the art \ac{DL} approaches for scene text \ac{OCR} are viable for the use case of
extracting textual label data from images taken in real world conditions?

The definition of the viability of an approach must be determined for this.
What qualities such as detecting alpha-numeric strings or suitability despite
inadequate image conditions must a solution have~\citep{ghosh_visual_2017, hu_gtc_2020}?

It is difficult to assess how well a \ac{DL} approach performs before it has been
implemented and tested on the specific problem or representative dataset~\citep{arpteg_software_2018}.
This justifies the need to create an overview rather than pointing out a single approach which is
deemed the most promissing.
In order to create the overview the necessary steps in the process of \ac{STS} need to be highlighted,
from localizing possible text instances to predicting the characters or
words~\citep{long_scene_2021, sourvanos_challenges_2018}.
The ways in which the respective issues for the steps are solved need to be
identified from literature, listed and explained alongside.

The article~\cite{ashmore_assuring_2021} defines four phases of the \ac{ML} lifecycle, namely,
Data Management, Model Learing, Model Verification and Model Deployment.
Only the substage Model Selection from Model Learning will only be looked at in the scope of this
thesis.
\cite{goodfellow_deep_2016} states:
`Nearly all deep learning algorithms can be described as particular instances of a fairly simple
recipe: combine a specification of a dataset, a cost function, an optimization procedure and a model.'
Other aspects such as data analysis, implementation, training, deployment and maintenance of a
solution in a production environment shall not be performed.
Based on this theses, further further verification, implementation and testing can then be performed.
The overview and subsequent analysis thereof creates a foundation for finding the right solution,
it does however not contain any claims about the degree of goodness or about the certainty of
solving the given problem.

% XXX: einarbeiten
\begin{comment}
Abgrenzung für Training NN
- no talk about Setup (preprocessing, weight initialization,regularization)
- No talk about training dynamics (learning rate,large batch training, hyperparameter)
- No after training ( transfer learning, model ensemble)
\end{comment}

\section{Methodology}\label{se:methodology}
The methodology of this thesis can be labeled as a literature review~\citep{snyder_literature_2019,
torraco_writing_2005}.
The goal is to provide an overview over current \ac{DL} techniques that can help in
choosing which to implement and to test to solve the specific problem of \ac{STS}, defined in
Section~\ref{se:problem} and more detailed in Chapter~\ref{ch:problem}.

The research question guiding the process is most crucial~\citep{snyder_literature_2019}:
Which state of the art \ac{DL} approaches for \ac{STS} are viable for the use case of extracting
textual label data from images.
In order to improve the validity for the subsequent analysis, the problem is dissected further.
This includes analysing the specific use case as well as researching which qualities have been
identified as generally critical for scene text approaches.
The qualities are taken from literature which covers \ac{ML} in general to literature
which covers \ac{OCR} under challenging scene text conditions.

For a literature review, it is important to report how the information was found and
synthesized~\citep{torraco_writing_2005}.
Therefore, each section in the overview contains a paragraph about said information.
Before getting into current research level, a foundation of knowledge about the field must be layed.
A taxonomy is created for this which is useful to classify and give context to innovations in the
field.
The partition of tasks and categorization of approaches is conducted according to information from
overview literature such as~\cite{long_scene_2021,chen_text_2021,cong_comparative_2019} and
difference of approaches that can be identified in research such
as~\cite{qiao_text_2021,sheng_centripetaltext_2021,liu_accurate_2020,deng_pixellink_2018}.
The taxonomy is determined according to the requirement for clarity of subsequent provision of
current research.

The strategy for researching current innovations is most important for a literature
review~\citep{snyder_literature_2019}.
This includes determining databases and keywords that are used, as well as exclusion criteria
that are enforced~\citep{torraco_writing_2005}.
Innovations are explored through searching in the Google Scholar database.
A criterion for further examination is an appropriate amount of citations for the piece of literature
in question.
Additionally, literature is selected through citations for and by literature which has already been
identified as important.
All research after 2018 which pertains to extracting scene text is regarded as relevant.
Standard \ac{OCR} solutions may not hold validity in practice, as the image and text conditions can
vary in the defined problem~\citep{chen_text_2021}.
An important criterion is that the paper contributes to the \ac{ML} model.
This extends to the whole pipeline from preprocessing an image to the final result of the model.
The identified literature is synthesized into an overview that is aligned according to the taxonomy.

In the analysis possibly viable approaches are compared with the required qualities defined
in Chapter~\ref{ch:problem}.
The comparison will be organized according the taxonomy which facilitates the clarity and
comprehensibility.
The comparison is arranged in hierarchical fashion: different pipeline categories,
different categories for pipeline tasks and innovations for those tasks.
The analysis thus shows which approaches are worthwhile to apply the whole \ac{ML} lifecycle to.

\section{Expected Results}
In addition to a deeper understanding of the problem and its detailed definition, the literature
review lays the foundation for finding the right approach for the extraction of textual
information from images with equipment labels through literature review.
In the subsequent analysis different approaches are highlighted for their theoretical fit as a solution.

In the following, the structure of this thesis is listed and each chapter's expected
result is detailed along with its benefits for the overall objective of producing an overview of
state of the art \ac{STS} relevant for the problem described in Section~\ref{se:problem}.
Chapter~\ref{ch:theoretical} lays the theoretical foundation for later chapters.
This includes general principles of \ac{DL} and by extension \ac{ML} but also of \ac{STS}.
In Chapter~\ref{ch:problem} the problem from Section~\ref{se:problem} is addressed in more detail.
The result shall be a firm understanding of qualities that a solution must possess.
These requirements are the point of focus for the further examination of techniques.
In Chapter~\ref{ch:research} current research in regards to the
identified requirements is examined.
The overview is twofold: a taxonomy for the pipeline and its approaches as well as innovations in
current research.
The resulting overview can be viewed as a basis for a decision when it comes implementing a practical
solution.
Therefore, it enables the discussion in Chapter~\ref{ch:discussion}.
Here not only the results and the availability of a solution but also the methodology of this work
is assessed critically.
The conclusion is a summary of the results compared to the expected results detailed in this chapter
as well as an outlook for further research into the topic.
