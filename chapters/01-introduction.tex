\chapter{Introduction}\label{ch:intro}
\section{Motivation}
\ac{OCR} is the concept of extracting typed, handwritten or printed text
from an image.
Techniques for this concept have improved a lot due to the advances in the field of
\ac{DL}~\citep{zhao_improving_2020}.
When compared to traditional methods \ac{DL} improves automation, effectiveness and
generalization~\citep{chen_text_2021}.
\ac{DL} is a technology based on \acp{NN} where data is processed
in multiple layers to extract complex features to solve a given problem~\citep{shrestha_review_2019}.
\ac{DL} has only caught on in the recent years as the big computational cost has been met
by improvement in computer hardware as well as in automatic feature
learning~\citep{ponti_everything_2017, chen_text_2021}.
Applying these new capabilities and finding the right solution in the space of \ac{DL} for the
use case of extracting information of labels is the focus of this thesis.
This is an interesting task as performance of \ac{OCR} systems in complex scenes is still
challenging~\citep{zhao_improving_2020}.
Such scenes entail natural scenes captured by a camera.
\ac{OCR} in these conditions is also known as \ac{STR}~\citep{chen_text_2021}.
Factors such as complex backgrounds, noise, perspective and variability in fonts, colors and sizes,
of scene texts complicate the process~\citep{hu_gtc_2020,chen_text_2021}.

\section{Problem description}\label{se:problem}
Technicians in the field work with different equipement.
It is useful to digitize the lables of such equipement, to keep an overview over the
inventory~\citep{abramowicz_business_2019}.
The goal of this thesis is to find a solution which simplifies the digitization of equipment labels.
The research question guiding the process is most crucial: Which state of the art \ac{DL}
approaches for \ac{STR} are viable for the use case of extracting textual label data from images.
The definition of the viability of an approach has to be determined for this.
What qualities such as detecting alpha-numeric strings or suitability despite
inadequate image conditions must a solution have~\citep{ghosh_visual_2017, hu_gtc_2020}?

It is difficult to assess how well a \ac{DL} approach performs before it has been
implemented and tested on the specific problem or dataset~\citep{arpteg_software_2018}.
Therefore, multiple promissing approaches that can be implemented and experimented with have to
be identified and analyzed.
The reasearch and discussion of techniques from end-to-end \ac{STR} to dividing the
process into text detection and text recognition is centered~\cite{chen_text_2021} around the
requirements which are given by the problem.

The article~\cite{ashmore_assuring_2021} defines four phases of the \ac{ML} lifecycle, namely,
Data Management, Model Learing, Model Verification and Model Deployment.
Only the substage Model Selection from Model Learning will only be looked at in the scope of this
thesis.
Other aspects such as data analysis, implementation, training, deployment and maintenance of a
solution in a production environment shall not be performed.

\section{Methodology}\label{se:methodology}
The methodology of this thesis can be labeled as a literature review~\citep{snyder_literature_2019,
torraco_writing_2005}.
The goal is to provide an overview over current \ac{DL} pipelines and models that can help in
choosing which to implement and test in order to solve the specific problem defined in
Section~\ref{se:problem} and more detailed in Chapter~\ref{ch:problem}.

The research question guiding the process is most crucial: Which state of the art \ac{DL}
approaches for \ac{STR} are viable for the use case of extracting textual label data from
images.
In order to improve the validity for the subsequent analysis, the problem is disected further.
This includes analysing the specific use case as well as researching which qualities have been
identified as generally critical for \ac{STR} systems.
% FIXME: include following:
The qualities are taken from literature which covers \ac{ML} in general to literature
which covers \ac{STR}.

The literature is identified through searching in reputable journals.
All research after 2017 which pertains to \ac{STR} is regarded as relevant.
\ac{OCR} solutions may not hold validity in practice, as the image qualities can vary in the
defined problem~\citep{chen_text_2021}.
An important criteria is that the paper contributes to the \ac{ML} model.
This extends to the whole pipeline  from preprocessing an image to the final result of the model.
Concludent to to the distinction in Section~\ref{se:problem}, contributions to other stages in the
\ac{ML} lifecycle are not examined.
Therefore, keywords for the search include: Deep Learning, Scene Text Recognition, Pipeline,
Preprocessing, End-to-end, Text Recognition, Text Detection, Text Segmentation.

% FIXME: more pipelines?
The identified literature is synthesized into an overview over the most common approaches for
\ac{STR}.
This includes listing important factors for \ac{DL} such as the number of parameters, or which
type of layers are used in order to achieve success.
The overview will be organized into the categories for the \ac{ML} pipeline, such as End-to-End
solutions as in~\cite{xing_convolutional_2019} or a split into Text Detection and Text Recognition
as in~\cite{yang_learning_2021, chen_improvement_2018}.

In the analysis possibly viable approaches are compared with the qualities defined
in Chapter~\ref{ch:problem}.
The approaches are analysed in detail in regards to commonalities as well as differences and the
possible effect on the feasibility.
The analysis thus shows which approaches are worthwhile to apply the whole \ac{ML} lifecycle to.

\section{Expected results}
In addition to a deeper understanding of the problem and its detailed definition, the literature
review lays the foundation for finding the right approach for the extraction of textual
information from images with equipment labels through literature review.
In the subsequent analysis different approaches are highlighted for their theoretical fit as a solution.

In the following, the structure of this thesis is listed and each chapter's expected
result is detailed along with its benefits for the overall objective of producing an overview of
state of the art \ac{STR} relevant for the problem described in Section~\ref{se:problem}.
comprehension of the following chapters is gathered.
This includes general principles of \ac{DL} and by extension \ac{ML} but also of \ac{OCR}.\@
In Chapter~\ref{ch:problem} the problem from Section~\ref{se:problem} is addressed in more detail.
The result shall be a firm understanding of qualities that a solution must possess.
These requirements are the point of focus for the further examination of \ac{STR} techniques.
After laying the foundation, in Chapter~\ref{ch:research} current research in regards to the
identified requirements is examined.
The resulting overview can be viewed as a basis for a decision when it comes implementing a practical
solution.
Therefore it enables the discussion in Chapter~\ref{ch:discussion}.
Here not only the results and the availability of a solution but also the methodology of this work
are assessed critically.
The conclusion is a summary of the results compared to the expected results detailed in this chapter
as well as an outlook for further research into the topic.
