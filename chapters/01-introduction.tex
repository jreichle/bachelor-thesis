\chapter{Introduction}\label{ch:intro}
\section{Motivation}
% TODO: make more inciting
Optical Character Recognition (OCR) is the concept of extracting typed, handwritten or printed text
from an image.
Techniques for this concept have improved a lot due to the advances in the field of Deep
Learning~\cite{zhao_improving_2020}.
Deep Learning is a technology based on Artificial Neural Networks where data is processed
in multiple layers to extract complex features and solve a given problem~\cite{shrestha_review_2019}.
Deep Learning has only caught on in the recent years as the big computational cost has been met
by the improvement in computer hardware~\cite{ponti_everything_2017}.
Finding the right solution in the space of deep learning and applying these new capabilities to
the use case of extracting information of equipment labels is the focus of this thesis.
% TODO: state relevance

\section{Problem description}\label{se:problem}
The central part of the bachelor thesis is laying a foundation for finding the right approach for
the extraction of textual information from images with equipment labels.
This includes defining functional requirements such as detecting rotated text but also non functional
requirements such as given computational power of mobile devices the solution.
These requirements define properties that a solution must have in order to be classified as viable.
Thus the discussion of techniques from end-to-end OCR to dividing the process into text detection and
text recognition is centered around the requirements which are given by the problem.
Critical factors such as the availability of data and the complexity of implementing and training a
Deep Learning model also need to be assessed.

Following aspects such as implementing, training, deploying and maintaining a solution in a
production environment shall not be subject of this thesis.

Make Pipeline the problem?

\section{Methodology}\label{se:methodology}
The methodology of this \the\arbeit\ can be described as a literature review.
As such, the research question guiding the process is most crucial: Which state of the art Deep
Learning approaches for OCR are viable for the use case of extracting textual label data from
images.
The following section describes how relevant literature is identified, analysed and synthesised.

Add: how was problem defined

Searching:
search strategy for specific review: $\rightarrow$ provide reasoning behind choices
\begin{itemize}
    \item search tearms
    \item databases
    \item inclusion \& exclusion criteria: year of publification, type of article, journal, research
        quality
\end{itemize}
write all decisions down, how is search documented
The research is then extended to existing practical solutions for similar practical problems as
well as proposed architectures from academic research.
documentation of search process and selection, asses quality of search process and selection

Analysis:
how is data prepared for analysis
analyse problem: find requirements
type of information that needs to be extracted: what works, why, not necesseraly how
literature review (state of the art)
how? top journals of last few years
Theoretical knowledge about the models as well as practical information about results for

Synthesis

after review:
discuss different approchoaches pros and cons for problem
methodogical limitations: practice is always different, \ldots

\section{Expected results}
In the following section the structure of the \the\arbeit\ is listed and each chapter's expected
result is detailed along with its benefits for the overall objective of producing an overview of
state of the art OCR relevant for the problem described in Section~\ref{se:problem}.

In Chapter~\ref{ch:theoretical} the theoretical foundation for the \the\arbeit\ which is needed for
comprehension of the following chapters is gathered.
This includes general principles of Deep Learning and by extension Machine Learning but also of OCR.

In Chapter~\ref{ch:problem} the problem from Section~\ref{se:problem} is addressed in more detail.
The result shall be a firm understanding of functional and non-functional requirements both on the
technical and the business process side.
These requirements are the point of focus for the further examination of OCR techniques.

After laying the foundation, in Chapter~\ref{ch:research} current research in regards to the
identified requirements is examined.
The resulting overview can be viewed as a basis for a decision when it comes implementing a practical
solution.

Therefore it enables the discussion in Chapter~\ref{ch:discussion}.
Here not only the results and the availability of a solution but also the methodology of this work
are assessed critically.

The conclusion is a summary of the results compared to the expected results detailed in this chapter
as well as an outlook for further research into the topic.


`A major challenge in developing DL systems is the difficulties in estimating the results before a system has been trained and tested'\cite{arpteg_software_2018}
