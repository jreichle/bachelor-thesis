\chapter{Discussion}\label{ch:discussion}
This chapter contains a comparison between the taxonomy categories that were introduced in
Section~\ref{se:taxonomy}.
The analysis is followed by reflecting on the methodology and the thesis' results.

\section{Analysis}\label{se:analysis}
The comparison in this section is synthesized from information found in literature and
some observations regarding recent innovations that are examined in Section~\ref{se:innovations}.
The analysis is structured similarly to the overview sections: first compare approach categories
for \ac{STD} and \ac{STR} and then move on towards \ac{STS}.
The approaches' compared aspects are the qualities that were identified as relevant in
Section~\ref{se:relevant-qualities}: appropriateness, performance in conjunction with robustness,
efficiency.
Note that appropriateness entails the subqualities derived from the use
case (see Table~\ref{tb:useCaseQualities}):
While the offline capability requirement is met by all of the approaches identified in this
thesis, semantics retention and alphanumeric recognition have to be analyzed further.
\ac{STD} is the relevant subtask for semantics retention and \ac{STR} for alphanumeric recognition.

The main challenges for \ac{STD} involve the tradeoff between speed and accuracy and
representing arbitrary shaped text instances~\citep{wang_efficient_2019}.
The innovations regarding \ac{STD} deal with multi-oriented and curved text (as can be seen in
Table~\ref{tb:STD-steps-properties}).
The innovations often include feature extraction and various segmentation innovations that improve
text representation and the subsequent text reconstruction, i.e., distinguishing text
instances.
\begin{table}[h]
    \centering\scriptsize
    \begin{tabular}{p{.10\textwidth}p{.10\textwidth}p{0.70\textwidth}}
        \multicolumn{2}{c}{\textbf{Approach category}} & \textbf{Shortcomings} \\
        \toprule
        Seg-free & & Curved text representation~\citep{long_scene_2021,wang_shape_2019} \\
        & & Linear anchor box bias with curved text~\citep{wang_shape_2019,ferrari_textsnake_2018} \\
        & & High aspect ratio test~\citep{shi_detecting_2017,long_scene_2021} \\
        & One-stage & More efficient, straightforward architecture~\citep{lu_mimicdet_2020} \\
        & Twe-stage & More accurate predictions due to refinement~\citep{lu_mimicdet_2020} \\
        \midrule
        Seg-based & & Separating different text instances~\citep{wang_shape_2019} \\
        & & Text instance construction is complex and computation
            intensive~\citep{xie_aggregation_2019,liao_real-time_2019,qiao_text_2021} \\
        & & Incorporate several computation intensive stages~\citep{dai_fused_2018} \\
        & & More vulnerable to noise~\citep{long_scene_2021} \\
        \bottomrule
    \end{tabular}
    \caption{STD approach category shortcomings\label{tb:STD-shortcomings}}
\end{table}

Segmentation-free or \ac{BB} regression-based approaches have trouble with curved text because of
the anchor boxes' linear bias~\citep{wang_shape_2019,ferrari_textsnake_2018}.
Additionally, the regressed \acp{BB} have trouble accurately representing the shape of curved
text instances~\citep{long_scene_2021,wang_shape_2019}.
Representing multi-oriented text is not a problem, on the other
hand~\citep{liao_textboxes_2018,jiang_r2cnn_2017}.
Long aspect ratios are another problem of scene text which degrades \ac{BB} regression
performance~\citep{shi_detecting_2017,long_scene_2021}.

On the other hand, segmentation-based approaches take advantage of the fact that every part of the
text instance can locally be verified as such~\citep{long_scene_2021}.
The bottom-up approach alleviates the problem of long aspect ratios~\citep{shi_detecting_2017}.
The segmentation-based approaches facilitate a more natural representation of curved
text~\citep{dai_fused_2018,long_scene_2021} (see Figure~\ref{fig:curved-text-representations}).

However, the segmentation pipeline is more computation-intensive, as it incorporates more complex
stages~\citep{dai_fused_2018} and complicated text instance
construction~\citep{xie_aggregation_2019,liao_real-time_2019,dai_fused_2018}.
The text instance reconstruction goes hand in hand with the separation of different instances.
This challenging task is vulnerable to noise~\citep{long_scene_2021} and focuses on
many innovations (see Figure~\ref{fig:curved-text-representations}).

When comparing \ac{BB} regression approaches, one-stage approaches are generally more
efficient  because of their straightforward architecture, while two-stage approaches can generally
predict the more accurately  because of the refinement process~\citep{lu_mimicdet_2020}.
Note that no meaningful comparison of sub-text level or pixel-level segmentation could be identified.
To what \ac{STD} is concerned, the bottom line seems to be:
Is curved text detection or efficiency more critical?
\begin{table}[h]
    \scriptsize\centering
    \begin{tabular}{llccc}
        \multicolumn{2}{c}{\textbf{Approach category}} & \textbf{Semantics retention}
                                                       & \textbf{Performance} & \textbf{Efficiency} \\
        \toprule
        Seg-free & & $\times$ & $\times$ & \checkmark\ \\
           & 1-stage & $\times$ & $\times$ & \checkmark\ \\
           & 2-stage & $\times$ & \checkmark\ & $\times$ \\
        \midrule
        Seg-based & & \checkmark\ & \checkmark\ & $\times$ \\
        \bottomrule
    \end{tabular}
    \caption{Curved STD approach category comparison\label{tb:STD-comparison}}
\end{table}
Table~\ref{tb:STD-comparison} shows the synthesized comparison for curved \ac{STD} based on their
shortcomings.
The table compares the categories seg-free and seg-based and the sub-categories 1-stage and 2-stage.

\ac{STR} research focuses on accurately recognizing 2d text instances.
The innovations found for \ac{STR} (see Table~\ref{tb:STR-steps-properties}) show that the research
mainly focuses on robustly recognizing curved text instances (rectification approaches and
2d-attention).
It can be noted that the innovative approaches are mostly attention-based.

However, attention also has shortcomings (see Figure~\ref{tb:STR-shortcomings}).
\begin{table}[h]
    \centering\scriptsize
    \begin{tabular}{p{.115\textwidth}p{.15\textwidth}p{0.66\textwidth}}
        \multicolumn{2}{c}{\textbf{Approach category}} & \textbf{Shortcomings} \\
        \toprule
        Seg-based & & Accurate detection of individual
                        characters~\citep{chen_text_2021,cheng_aon_2018} \\
        & & Disregard for contextual information between characters~\citep{chen_text_2021} \\
        & & Incorporate several computation intensive stages~\citep{liu_abcnet_2020} \\
        \midrule
        \ac{EnDe}-based & & Curved and multi-oriented text~\citep{cheng_aon_2018, long_scene_2021} \\
        & CTC-based & Prone to overfitting~\citep{chen_text_2021} \\
        & & Isolated word recognition~\citep{cong_comparative_2019} \\
        & & Not applicable to recognition of 2d text instance~\citep{cheng_focusing_2017,
            xie_aggregation_2019,chen_text_2021} \\
        & Attention-based & Sentence and long sequence
            recognition~\citep{cong_comparative_2019,chen_text_2021} \\
        & & Problems without vocabulary~\citep{wan_vocabulary_2020} \\
        & & Attention drift leads to missing or superfluous
            characters~\citep{liao_scene_2018,xie_aggregation_2019,chen_text_2021}\\
        & & Adapted 2d-attention requires a lot of storage and
            computation~\citep{xie_aggregation_2019,chen_text_2021} \\
        \bottomrule
    \end{tabular}
    \caption{STR approach category shortcomings\label{tb:STR-shortcomings}}
\end{table}
For segmentation-based approaches, the main shortcomings are as follows.
Localizing individual characters is computationally expensive~\citep{zhan_esir_2019}.
A complex pipeline is needed to predict characters and align them~\citep{liu_abcnet_2020}.
Take~\cite{wan_textscanner_2020}, for example, which leverages a separate geometry branch
to help with word formation with the predicted characters from the classifying branch.
Segmentation is also susceptible to errors~\citep{zhan_esir_2019,cheng_aon_2018,chen_text_2021}.
The nature of scene text reinforces the errors: complex backgrounds, noise, perspective,
and other factors~\citep{hu_gtc_2020,chen_text_2021,baek_what_2019}.
Additionally, segmentation-based approaches cannot use contextual information between
characters~\citep{chen_text_2021}.

\ac{EnDe}-based \ac{STR}, on the other hand, is modeled to take contextual information into
account~\citep{long_scene_2021,chen_text_2021}.
The problem manifests itself in dealing with 2d-text
instances~\citep{long_scene_2021,liao_scene_2018}.
The \ac{EnDe} basic approaches use decoders that work with sequences (and therefore
1d)~\citep{long_scene_2021,cheng_aon_2018}.
The introduced innovations deal with this in two ways: rectify 2d text into 1d
text~\citep{zhan_esir_2019,luo_multi-object_2019,shi_aster_2019,liu_char-net_2018}, adapt the
attention mechanism to 2d input~\citep{li_show_2019}.
Note that \ac{CTC} cannot be adapted to 2d input~\citep{cheng_focusing_2017,xie_aggregation_2019},
and the attention adaption by~\cite{li_show_2019} is memory and
computation-intensive~\citep{xie_aggregation_2019}.

When comparing \ac{CTC}-based to attention-based approaches, it can be noted that \ac{CTC} deals
better with long text instances or sentences while attention is better with singular
words~\citep{cong_comparative_2019,chen_text_2021}.
This is because the misalignment in attention can cause missing or superfluous characters (attention
drift)~\citep{bai_edit_2018,liao_scene_2018,cheng_focusing_2017}.
Attention-based approaches have the upper hand when incorporating implicit language models
or lexicons, but \ac{CTC} performs better without language priors~\citep{cong_comparative_2019}.
This is especially important because the text in the use case under consideration contains
alphanumeric strings which are not part of any lexicon.
When it comes to computational efficiency, both \ac{CTC} and attention are expensive and
time-consuming~\citep{chen_text_2021}.
For \ac{STR} \ac{CTC} seems to be the choice for efficiency and attention for robustness for
curved text.
\begin{table}[h]
    \scriptsize\centering
    \begin{tabular}{llccc}
        \multicolumn{2}{c}{\textbf{Approach category}} & \textbf{Alphanumeric recognition}
                                                       & \textbf{Performance} & \textbf{Efficiency} \\
        \toprule
        Seg-based & & \checkmark\ & $\times$ & $\times$ \\
        \midrule
        EnDe-based & & \checkmark\ & \checkmark\ & \checkmark\ \\
           & CTC-based & \checkmark\ & $\times$ & \checkmark\ \\
           & Attention-based & $\times$ & \checkmark\ & $\times$ \\
        \bottomrule
    \end{tabular}
    \caption{Curved STR approach category comparison\label{tb:STR-comparison}}
\end{table}
Table~\ref{tb:STR-comparison} shows the synthesized comparison for curved \ac{STR} based on their
shortcomings.
The table compares the categories seg-based and EnDe-based and the sub-categories CTC-based and
attention-based.

2-stage approaches have the upper hand when it comes to end-to-end \ac{STS}.
These approaches are in focus for research because they have crucial
advantages (see Table~\ref{tb:E2E-shortcomings})~\citep{chen_text_2021}.
\begin{table}[h]
    \centering\scriptsize%
    \begin{tabular}{p{.11\textwidth}p{.70\textwidth}}
        \textbf{Category} & \textbf{Shortcomings} \\
        \toprule
        2-step & Error propagation between detection and
            recognition~\citep{chen_text_2021,long_scene_2021}\\
        & No joint optimization~\citep{qiao_text_2021, chen_text_2021}\\
        & Computation requirements for two feature extraction
            CNNs~\citep{liu_fots_2018,chen_text_2021} \\
        2-stage & --- \\
        \bottomrule
    \end{tabular}
    \caption{STS approach category shortcomings\label{tb:E2E-shortcomings}}
\end{table}

2-stage approaches are more efficient than 2-step approaches because they do not compute feature maps
twice~\citep{liu_fots_2018,chen_text_2021}.
This carries much weight since feature extraction is usually the most time and
computation-consuming step~\citep{liu_fots_2018}.
The direct combination of \ac{STD} and \ac{STR} also impacts the performance.
Because of the connection, both stages are fully differentiable and can be jointly
optimized~\citep{chen_text_2021,long_scene_2021,qiao_text_2021}.
Without joint optimization, errors in the detection step can be propagated to the
recognition, resulting in performance degradation~\citep{chen_text_2021,qiao_text_2021}.
\begin{table}[h]
    \scriptsize\centering
    \begin{tabular}{lcccc}
        \textbf{Category} & \textbf{Performance} & \textbf{Efficiency} \\
        \toprule
        2-step & \checkmark\ & $\times$ \\
        2-stage & \checkmark\ & \checkmark\ \\
        \bottomrule
    \end{tabular}
    \caption{Curved STS approach category comparison\label{tb:STS-comparison}}
\end{table}
The resulting comparison between the 2-step and 2-stage categories concerning curved \ac{STS} can be
found in Table~\ref{tb:STS-comparison}.

\section{Reflection}
This thesis aimed to create an overview of \ac{STS} approaches to facilitate
choosing a practical problem.
A couple of factors impede this goal.
The first is that all entities of a \ac{MLS} need to be examined to develop a proper
solution~\citep{siebert_construction_2021,nakamichi_requirements-driven_2020}.
Because not all entities (most notably the data entity) are available, this was not possible.

Another reason is the chosen cutoff point of 100 citations as a requirement for impactful literature.
Even though it is helpful to filter for the most groundbreaking advancements in the field,
newer innovations that have not been revealed long enough to earn enough citations might be left
out of the equation.

The analysis does not contain quantitative data on performance or efficiency regarding
singular approaches or even task categories.
This is because the literature on \ac{ML} and \ac{DL} benchmarking suggests that it is not good
to combine and compare results from different
studies~\citep{baek_what_2019,arpteg_software_2018,long_scene_2021}.
There are different rationales for this.
Because different studies incorporate different components such as hardware, platform, source code,
or configuration, the comparison would be flawed~\citep{arpteg_software_2018,baek_what_2019}.
Additionally, different studies might present a different interpretation of
metrics~\citep{long_scene_2021}.
Another reason is the benchmark datasets: datasets are used inconsistently in
many cases.
Often specific images are left out for various reasons~\citep{baek_what_2019}.
