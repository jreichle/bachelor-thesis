\chapter{Conclusion}
This thesis facilitates a comprehensive overview of techniques that are used for \ac{STS}.
The associated tasks and the different approaches help compare different
solutions for the identified qualified that are important for the use case
(appropriateness, performance/robustness, efficiency).

The overview and literature review reveal that \ac{STD} and \ac{STR} are concerned
with robust performance for multi-oriented and curved text instances.
For \ac{STD}, the more efficient \ac{BB} regression based methods can detect multi-oriented text well.
Because of their representational deficiencies, pixel level segmentation based methods are better for
detecting curved text.
For the semantics retention subquality for appropriateness, it is, therefore, essential to consider
the text properties of the dataset.
The attention mechanism has become the primary approach for \ac{STR} robustness.
However, many innovations are concerned with the curved text rectification stage that both
\ac{CTC} and attention based \ac{EnDe} solutions share.
\ac{CTC} has the advantage over the attention mechanism for recognition of alphanumeric strings.
Because of their remarkable efficiency and competitive performance, 2 stage approaches focus
on \ac{STS} research.
The efficient combination of \ac{STD} and \ac{STR} stages and reuse of feature maps are topics
of the innovations in the field.

For future work, it would be helpful to consider different steps in the \ac{ML}
lifecycle~\citep{watanabe_preliminary_2019} or build a \ac{MLS} around the identified \ac{DL}
approach for \ac{STS}~\citep{siebert_construction_2021,nakamichi_requirements-driven_2020}.
An example would be to design a supervision mechanism for model monitoring that is important for
\ac{DL} in production systems~\cite{nakamichi_requirements-driven_2020,watanabe_preliminary_2019}.
Another possible step would be to design and carry out a comprehensive study to generate quantitative
data regarding the qualities that a \ac{STS} solution must possess.
