\chapter{Problem analysis}\label{ch:problem}
This chapter entails an analysis of the problem which is the research question's foundation.
It is crucial, as the quality of requirements ultimately determines the quality of the overview and
subsequent analysis.

Requirements for a software system that involves \ac{ML} and thus \ac{DL} requirements differs from
the traditional approach. The data-driven software components are not entirely defined by the programmer
but are influenced by data.
The system acts with dependency on the test data~\citep{siebert_construction_2021}.
This poses a challenge in determining requirements and measuring quality of
results~\citep{nakamichi_requirements-driven_2020}.
Instead of categorizing functional and non-functional requirements, like for traditional
software projects~\citep{zowghi_requirements_2014}, qualities that a \ac{MLS} must possess
are defined as proposed.
In~\cite{nakamichi_requirements-driven_2020,siebert_construction_2021} systematic approaches for
identification and documentation are detailed.
In \acp{MLS} various entities interact to in order to produce the desired functionality.
\cite{nakamichi_requirements-driven_2020} suggests that in order to adequately evaluate the qualities,
it is vital to not only consider the model but the entire \ac{MLS}.
These entities are data, model, infrastrucure, environment and platform (also system but as explained
shortly, extra granularity would be redundant)~\citep{nakamichi_requirements-driven_2020,
siebert_construction_2021}.

For this thesis only the model can be assessed because neither of the other entities can be regarded
as given.
% FIXME: missing data most problematic (CRISP-DM for siebert)
The model consists of subcomponents organized in directed acyclc graph building a
pipeline~\citep{siebert_construction_2021}.
This directed acyclic graph depicts everything from processing the images to the extracted
information~\citep{siebert_construction_2021}.
That is why the systematic approaches cannot be performed in the scope of this thesis.
Instead many qualities that are highlighted by research that fit the problem are taken into account
along with two critical qualities (alphanumeric recognition, semantic retention) that are directly
derived from the use case.

When it comes to documenting the identified qualities,
both~\cite{nakamichi_requirements-driven_2020} and~\cite{siebert_construction_2021} define a meta
model for qualities that combines qualities with
measurement methods and values and assignes them to an entity of the \ac{MLS}.
The implementation and testing phase are not performed in the scope of this thesis and the
difficulty in assessing the performance ahead of those phases, prevents complicates the evaluation
of measurements.
Additionally, experimental results from literature can only be compared as long as factors such as
hardware, platform, source code, configuration and dataset are uniform~\citep{arpteg_software_2018}.
This applies to studies that create an overview such as~\cite{chen_text_2021,long_scene_2021}.
These studies can only be regarded as guiding values.
That's why measurements are not defined, as evaluation would only deliver a false
sense of certainty.
% FIXME: explain systematic approach of siebert??

The problem can be depicted by a use case.
This use case sets the foundation for determining requirements for an
approach because qualities derive from the intended purpose of
use~\citep{siebert_construction_2021}.
For this thesis, the basic use case is as follows:
% TODO: where exactly
A technician takes a photo of a device label with his smart phone.
The resulting image contains printed textual information which must be extraced by an application on
the smart phone.
Space and structure of this information can vary from label to label (see figure~\ref{fig:examples}).
\begin{figure}[h]
    \centering
    \subfigure[Positive example\label{fig:good-example}]{\includegraphics[width=0.40\textwidth]
        {img/Image-Example-Positive.jpg}}
    \subfigure[Negative example\label{fig:bad-example}]{\includegraphics[width=0.40\textwidth]
        {img/Image-Example-Negative.jpg}}
    \caption{Examples for label images\label{fig:examples}}
\end{figure}
The text, spacing and structure carries semantic information which can be important for later
processing in the scope of a business process~\citep{chen_text_2021}.
The goal is to extract the text and preserve semantics from structure and space.
This means text and the respective coordinates, height, width and a possible rotation angle must
be output as the result~\citep{yang_learning_2021}.
Those values can then be transformed into other formats such as JSON or HTML as needed.
The labels can contain arbitrary alpha-numeric strings such as serial numbers (see
figure~\ref{fig:examples}).
This results in the requirement that the \ac{DL} model has to be able to recognize sequences that
are not part of a predefined lexicon~\citep{ghosh_visual_2017}.
% FIXME: justify following
The qualities for the \ac{MLS} that can be derived directly from the use case (see
table~\ref{tb:literatureQualities}) can be regarded as excluding criterias.
\begin{table}[h]
    \centering
    \begin{tabular}{c l}
        Alphanumeric recognition & Recognize alphanumeric strings such as serial numbers \\
        Semantics retention & Retain semantics given implicitly be space and strucure \\
                            & of text in labels \\
    \end{tabular}
    \caption{Qualities specific to use case --- exclusion criterias}
\end{table}

In addition to the qualities that arise directly from the use case, literature reveals a number of
common qualities in regards to \ac{MLS} (see table~\ref{tb:literatureQualities}), some of which
can be regarded as relevant and other do not hold any importance for the specific use case.
\begin{table}[h]\label{tb:literatureQualities}
    \centering
    \begin{tabular}{c c}
        Imporant Qualites & Unimportant \\
        \hline
        Performance & Interpretability \\
        Robustness & Reusability \\
        Maintainability & Fairness \\
        Performance efficiency & Security / Data protection \\
    \end{tabular}
    \caption{Qualities identified by literature}
\end{table}
\textbf{here important qualities explained (example: robustness)}
% Performance
% FIXME: text performance vs generalization, close to robustness!
% Robustness
Due to the uncontrolled environment of \ac{STR} in the practical aspect of taking the images on-site
beneficial image properties can not be guaranteed~\citep{chen_text_2021}.
Robust text extraction can be influenced by factors such as complex backgrounds, text form
(text rotation, font variability, arrangement), image noise (lighting conditions, blur,
interference and low resolution) and access (perspective, shape of
text)~\citep{oyedotun_deep_2015,ghosh_visual_2017,chen_text_2021}.
Therefore, these properties have to be accounted for when determining the viability for an approach.
An example for bad image quality in regards to \ac{OCR} can be seen in figure~\ref{fig:bad-example}.
