%%%%%%%%%%%%%%%%%%%%%%%%%%%%%%%
% Tradeoff from problem chapter
%%%%%%%%%%%%%%%%%%%%%%%%%%%%%%%
\section{Tradeoff}
Note: tradeoff between accuracy and computational cost $\rightarrow$ mobile phone dilemma

When determining whether automation is an improvement four aspects have to be examined.
These are time, costs, quality and flexibility.
The aspects build a quadrangle that is based on the optimizing trade-off between the
factors~\cite{dumas_fundamentals_2013}.

Without software supporting the task of reading the name of the picture and typing it into
the system, can take long seconds, whereas a trained \ac{DL} model could complete the task
in a mere instant.
Therefor automisation via \ac{DL} should improve the efficiency of the process when compared to
manually reading and typing the information off the image.

Training costs for a \ac{DL} model are very high due to the computing intensive
backpropagation algorithm that tunes the network to the data.
But the usage cost is low.
For manual labor the opposite is the case as training a person to type in a label is done quickly
and labor costs are high in comparison to the expenses for running the model.

Both \ac{DL} models and human labor are not 100\% accurate.
The question is whether the model can be as accurate or even better than its human counterpart.
This is especially interesting when it is applied in the real world where it might have to do good
in subpar situations.
An example is bad image quality.

Flexibility is concerned with how well a process can adjust to changing requirements.
A set of new equipment names that have to be included can pose a problem to a \ac{DL} model
because it is not trained for the new data.
A human on the other hand should not have any problems in this regard.

The main concern for the solution's efficacy is whether it is accurate enough.
Therefor this work focuses on this aspect in particular.
