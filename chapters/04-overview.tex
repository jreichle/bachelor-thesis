\chapter{Current Research}\label{ch:research}
\section{Different pipeline Frameworks}
What is wrong with scene text recognition model comparison~\cite{baek_what_2019}
Four stages derived from existing STR-Models
\begin{itemize}
    \item Transformation: normalize input image $\rightarrow$ Spatial Transormer Network
        \begin{itemize}
            \item transform input image $X$ into $\tilde{X}$
            \item if curved and tilted texts / other diverse shapes are forwarded unaltered,
                feature extraction needs to learn an invariant representation
            \item thin-plate spline transformatino (variant of spatial transformation network)
                $\rightarrow$ smooth spline interpolation between set of fiducial points
        \end{itemize}
    \item Feature extraction: map input image to representation that focuses on relevant attributes,
        while suppressing irrelevant features
        \begin{itemize}
            \item use \ac{CNN} to abstract image $\tilde{X}$ to output visual feature map
                $V=\{v_i\}, i=i,\ldots,I$\\
                I is number of columns in feature map, each column has a corresponding distinguishable
                receptive field along the horizontal line of $\tilde{X}$
            \item important architectures: VGG, RCNN, ResNet
        \end{itemize}
    \item Sequence Modeling: capture contextual information within sequence of characters
        \begin{itemize}
            \item reshape extracted featured to be a sequence of features $V$ (each column)
            \item use Bidirectional LSTM $\rightarrow$ sequence $H=Seq.(V)$
            \item this stage is optional
        \end{itemize}
    \item Prediction: estimate output character sequence
        \begin{itemize}
            \item use $H$ to predict a sequence of characters $Y=y_1,y_2,\ldots,y_n$
            \item options: Connectionist temporal classification or attention-based sequence prediction
        \end{itemize}
\end{itemize}
Tradeoff:
\begin{itemize}
    \item accuracy-speed
    \item accuracy-memory
\end{itemize}

Text Recognition in the Wild: A Survey~\citep{chen_text_2021}
\begin{itemize}
    \item pipeline
        \begin{itemize}
            \item text detection: text localization \& text verification
            \item (Text Segmentation)
            \item Text Recognition
        \end{itemize}
    \item various stages of \ac{OCR}:
        \begin{itemize}
            \item text localization: localize text components, group into candidate text regions with
                as little background as possible, DNN
            \item text verification: verify text candidate regions as text or non-text,
                filter false-positives, CNN
            \item text detection: determine whether text is present using localization and verification
                procedures, basis for end-to-end, can be regression or segmentation based
            \item text segmentation: most challenging, includes text line (splitting a region of multiple
                text lines into subregion of single text lines) and character segmentation (separating
                text instance into single characters, typically used in earlier approaches)
            \item text recognition: translates cropped text instance image into target string sequence,
                basis for end-to-end, DL encoder-decoder frameworks
            \item end-to-end-system: given scene text image $\rightarrow$ convert all text regions into
                target string sequences, includes detectoin, recognition and postprocessing, can be
                seen as indipendent subproblems but also joint by sharing information
        \end{itemize}
    \item text enhancement: recover degraded text, improve text resolution, remove distortions,
        remove background $\rightarrow$ reduce difficulty of recognition
\end{itemize}

Challenges in input preprocessing for mobile OCR applications~\cite{sourvanos_challenges_2018}
\begin{itemize}
    \item Acquistion: obtaining image --- digitization, binarization, compression
    \item Preprocessing: enhancing image quality --- noise removal, skew removal, thinning,
        morphological operations
    \item Segmentation: separating structural elements --- implicit and explicit segmentation
    \item Feature extraction: generating salient features --- geometrical, statistical
    \item Classification: categorizing individual characters to their respective classes --- clustering,
        neural networks, bayesian models, etc.
    \item Post-processing: improving and filtering --- contextual approaches, multiple classifiers, dictionary based approaches
\end{itemize}

Some part for which datasets have which characteristics? see \cite{long_scene_2021}

\section{Techniques/Modules for Improvement}

model pruning:~\cite{niu_26ms_2019}
integer inference:~\cite{ignatov_ai_2019}
