\erklaerung%

\begin{abstract}
    What Deep Learning approaches for Scene Text Spotting are there, and what are their shortcomings
    concerning the problem of extracting textual label data from images taken in real-world
    conditions?
    That is the question guiding this thesis.
    It is essential, as automized digitization has many economic upsides.
    A taxonomy is created that distinguishes different categories of approaches for clarity for
    subsequent chapters:
    Scene Text Spotting can be subdivided into Scene Text Detection and Scene Text Recognition.
    Both subtasks have different categories of approaches that are examined.
    A literature review is conducted to identify the most important innovations and gain insight
    into what shortcomings the field works on and what techniques are used in the process.
    The most prevalent issue is dealing with curved text robustly.
    For detection segmentation-based approaches and for recognition, attention-based approaches
    improve curved text recognition.
    Before analyzing the identified approaches, qualities are identified that serve as criteria
    for the comparison: appropriateness, performance/robustness, and efficiency.
    The analysis uses these qualities to compare the different categories.
    Depending on the specific dataset's properties, different approach categories can shine:
    regression-based detection efficiently deals with oriented text while segmentation-based
    detection is needed for curved text. For recognition, sequence-based approaches are the
    favorite.
    These entail CTC-based approaches for oriented text that contains longer text instances or
    alphanumeric strings and attention-based approaches for robustness for curved text.
    Additionally, CTC is the better choice for efficiency.
    For the whole task of Scene Text Spotting, the end-to-end differentiable 2-stage approaches
    that combine detection and recognition by sampling features are the clear favorite over 2-step
    approaches.

    \vspace*{70px}

    {\noindent
        \textbf{Keywords:} Deep Learning, Scene Text Spotting, Literature Review
    }
\end{abstract}

\tableofcontents

\newpage
\clearpage
\listoffigures

% delete group thing to have tables on new page

\newpage
\clearpage
\listoftables
\begin{comment}
\begingroup
    \let\clearpage\relax
    \listoftables
\endgroup
\end{comment}

\chapter*{Abbreviations}
\addcontentsline{toc}{chapter}{Abbreviations}
\printacronyms[heading=none]

\newpage
\clearpage
\renewcommand{\nomname}{Notation}
\addcontentsline{toc}{chapter}{Notation}
\printnomenclature%

%%
%% Nomenclature
%%

\nomenclature[C]{$\frac{dy}{dx}$}{Derivative of y with respect to x}
\nomenclature[C]{$\frac{\delta y}{\delta x}$}{Partial derivative of y with respect to x}
\nomenclature[C]{$\nabla_\x y$ or $\frac{\delta y}{\delta \x}$}{
    Gradient of y with respect to \textbf{x}
}
\nomenclature[C]{$\frac{\delta \y}{
    \delta \x}$}{Jacobian matrix $\textbf{J}\in\R^{n\times m}$ of$f:\R^n \rightarrow\R^m$
}


\nomenclature[D]{$\xti$}{The i-th example (input) from a dataset}
\nomenclature[D]{$\yti$ or $\textbf{y}^{(i)}$}{The target associated with $\xti$}
\nomenclature[D]{$\Xt$}{A set of training examples}
\nomenclature[D]{X}{The $m\times n$ matrix with input example $\xti$ in row $X_{u,:}$}


\nomenclature[O]{$\R$}{Real numbers}
\nomenclature[O]{$\norm{\x}$}{$L^2$ norm of \x}
\nomenclature[O]{$f(\x;\T)$}{A function of \x\ parametrized by $\T$}

\nomenclature[M]{\($a$\)}{Scalar}
\nomenclature[M]{\textbf{a}}{Vector}
\nomenclature[M]{$A$}{Matrix}
\nomenclature[M]{\textbf{A}}{Tensor}
\nomenclature[M]{$\textbf{a}^T$ or $A^T$}{Transposed vector or matrix}
