%%1%%%%%%%%%%%%%%%%
% introduction chaper
%%%%%%%%%%%%%%%%%%

%%%%%%%%%%%%%%%%%%
% methodology section

find 5 possible solutions and compare

Add: how was problem defined
only after (incl.) 2017

Searching:
search strategy for specific review: $\rightarrow$ provide reasoning behind choices
\begin{itemize}
    \item search tearms
    \item databases
    \item inclusion \& exclusion criteria: year of publification, type of article, journal, research
        quality
\end{itemize}
write all decisions down, how is search documented
The research is then extended to existing practical solutions for similar practical problems as
well as proposed architectures from academic research.
Documentation of search process and selection, asses quality of search process and selection

Analysis:
how is data prepared for analysis
analyse problem: find requirements
type of information that needs to be extracted: what works, why, not necesseraly how
literature review (state of the art)
how? top journals of last few years
Theoretical knowledge about the models as well as practical information about results for

Synthesis

after review:
discuss different approchoaches pros and cons for problem
methodogical limitations: practice is always different, \ldots


Introduction to Design Science~\citep{johannesson_introduction_2021}
Design Science (design and develop artifacts, contextual knowledge about artifacts) <-> empirical science (describe, explain, predict)

Define Artifact
- define problem to solve: current state --- destination  state
- Adress stakeholders?
- Functional Requirements
- Bin-functional requirements
- Artifact structure-> inner workings
- Take environment into account (side effects)

How to evaluate results?
- Find similar work
Which kind of contribution
- Improvement
- Routine Design
- Exaptation
- Invention

Knowledge types -> purpose
- Definitional knowledge: provides basic concepts required to express knowledge -> define concepts, without claims of existence
- Descriptive knowledge: describes, summarizes, generalizes, classifies observations-> describe, without claim of explanation or prediction
- Explanatory knowledge: answers how objects behave and why events occur- often in form of cause effect chains -> explain past, without claiming predictions
- Predictive knowledge: predict outcomes based on underlying factors without explaining causal relationship -> predict, not explain
- Explanatory and predictive knowledge
- Prescriptive knowledge: models and methods that help solve practical problems
Knowledge forms

Types of artifacts
- Constructs
- Models
- Methods
- Instantiations

Design theory: generate knowledge about produced artifact
- Purpose and scope
- Constructs
- Principle of form and function -> abstract blueprint or architecture that describes architecture
- Artifact mutability -> changes in state
- Testable propositions -> propositions about instantiation
- Justificatory knowledge -> knowledge that provides justification for design
- Principles of implementation
- Expository instantiation

Research strategy: Case Study
- Investigates multiple factors, events, relationships that occur in a real world case
- Characterization
    - Focus on one instance
    - Focus on depths
    - Natural setting
    - Relationships and processes
    - Multiple sources and methods
- Can be exploratory, descriptive, explanatory
Research strategy: action research
- Address practical problems
- Characteristics
    - Focus on practice
    - Change in practice
    - Active practitioner participation
    - Cyclical process: diagnosis, planning, intervention, reflection
    - Action outcomes and research outcomes
- Challenge: generalize results, remaining impartial
Research strategy	: Simulation
Research strategy	: mathematical and logical proof

Data collection
- Quantitative (numeric) - qualitative (text, …)
- Mixed approach -> use another strategy to verify results of first
- Strategies
    - Observation: directly observe phenomena, problem -> subjectivity -> do observation schedule (S. 165)
    - Documents: academic stuff

Data analysis
- Quantitative (numbers) - qualitative (text, …)
- Quantitative data: categorical, ordinal, interval, ratio

Deduce: specific conclusion from general principle
Induce: general principle from specific observations


Result: which knowledge type which knowledge form
Read research paper for clues for introduction

%%%%%%%%%%%%%%%%%%
% problem chaper
%%%%%%%%%%%%%%%%%%
Towards Requirements Specification for \ldots~\cite{hu_towards_2020}
requirements that ensure robustness (handle stressful environmental conditions and unseen or
unexpected data) are crucial

Definition of transformations: modifications in images
--- affine (e.g.\ scaling, rotation), perceptual context transformations (e.g.\ light sources, viewpoint)
$\rightarrow$ invariant (not changing e.g. class label) --- equivariant (e.g. bounding
                box position) requirements


Assuring the Machine Learning Lifecycle~\cite{ashmore_assuring_2021}
Activities in MLC Model Learning
\begin{itemize}
    \item Model Selection: decide model type, variant, structure of model
    \item Training
    \item Hyperparameter Selection
    \item Transfer Learning
\end{itemize}
desired properties for ML component: performance, robustness (handle stressful environmental
conditions and unseen or unexpected data), reusability, interpretability
`Comparing models is not always straightforward, with different models showing superior performance
against different measures. Composite metrics [59, 158] allow for a tradeoff between measures
during the training process.'

Requirements Engineering Challengs in Building AI-Based Complex Systems~\cite{belani_requirements_2019}
RE activities according to different AI-entities: data, model, system (for thesis only model)
Look into: Agent-based software engineering (see sources 11, 12 in paper)

Construction of a quality model for machine learning systems~\cite{siebert_construction_2021}
Data-driven software components: function not entirely defined by programmer but is derived from data
At the core of dd software component lies the notion of a model (by definition a
re-presentation/simplification of some part of reality)
Behavior of such components is fundamentally different from traditional software: input-output
relationship is usually non-linear $\rightarrow$ small discrepancy in input big discrepancy in output
Relationship only defined for test data $\rightarrow$ uncertainty in outcomes for unseen data
To build a quality model:
\begin{itemize}
    \item first define the usage scenarios
    \item define quality properties
    \item define quantity properties and how to measure
\end{itemize}
Different quality classifications:
\begin{itemize}
    \item service quality, product quality, platform quality
    \item data integrity, model robustness, system quality, process agility and costumer expectation
\end{itemize}
Different views/entities of the system: data, model, platform, environment
Qualiy model construction process (performed iteratively)
\begin{enumerate}
    \item Define quality meta-model: basic structure for documenting all quality properties and
        measures for quantifying those properties
        for model sse fig3
    \item Define use case and application context
    \item identify relevant ML quality requirements (for thesis only model part)
        \begin{itemize}
            \item usually systematic approach after CRISP-DM!!!
            \item Important aspects: data understanding, modeling
        \end{itemize}
    \item Idenfify relevant entities of ML System: data, model, \ldots\\
        Model view, specifically model type not instanciation (Data, System, Infrastructure,
        Enironment views not looked at in thesis)
        \begin{itemize}
            \item ML component consists of subcomponents organized in a directed acyclic graph/pipeline
        \end{itemize}
    \item Identify reference elements of an ML quality model: table of reference for requirements
        and entities
    \item instantiate quality model for use case\\
        \begin{itemize}
            \item interested in quality (property) at each step of the pipeline (entity)
            \item pipeline modeled as ahierarchy of entities, on lower levels, requirements for steps
                can be found
        \end{itemize}
\end{enumerate}
Important factors: context and user must be clear; iterative approach;

Maintainability:
encapsulation and modularity have to be rethought
changing anything changes everything

Requirements-Driven Methode to Determine Quality\ldots~\cite{nakamichi_requirements-driven_2020}
MLS is different from conventional software systems --- depends on amount and distribution of
training data in model learning and input data during operation $\rightarrow$ major challenge in
quality assurance

software systems [12] [21]. Sculley et al.\ pointed out that machine learning component is only a

small portion of the entire \ac{MLS}. Therefore, it is necessary to consider the entire MLS instead
of machine learning compo- nent [21]. This raises new problems in software engineering [1] [15].'

Research on quality problems still in early stage
More of a systemeatic approach in this paper
MLS parts and their issues
\begin{itemize}
    \item environmant/user
    \item system/infrastrucure: scope compliance, training/execute performance efficiency, output
        supervision, \ldots
    \item model
        \begin{itemize}
            \item approriatenss: is model appropriate for task and the required data
            \item performance training --- runtime
            \item \ldots
        \end{itemize}
    \item data: representativeness, balancedness, timeliness
\end{itemize}
quality meta model: quality model, quality measurement, environment
\begin{itemize}
    \item functional suitability: correctness (accuracy), completeness (generalization)
    \item reliability: maturity (robustness against change of input data and against noise data),
        availability (appropriatenss of operation maintenance)
    \item performance efficiency: time behaviour, resource utilization
    \item maintainability: modifiability (software, resource update), analyzability
    \item \ldots
\end{itemize}
