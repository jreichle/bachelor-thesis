\documentclass[pdftex,12pt,a4paper]{report}
\usepackage{dbstmpl}
\usepackage{subfigure}

% Hier die eigenen Daten eintragen
\global\arbeit{Bachelor Thesis}
\global\titel{Label Extraction from Image via Deep Learning}
\global\bearbeiter{Johannes Reichle}
\global\matrikel{04797218}
\global\betreuer{Prof.\ Dr.\ Rainer Schmidt}
\global\aufgabensteller{Random Ltd.}
\global\abgabetermin{XX.XX.2022}
\global\ort{Munich}
\global\fach{Information Systems and Management}

\begin{document}

\deckblatt

\erklaerung

\begin{abstract}
    Here abstract for \the\arbeit.
\end{abstract}


\tableofcontents

\chapter{Introduction}\label{ch:intro}

% TODO: replace Computer Vision with OCR -> make sure it's ok with citations
\section{Motivation}
Nowadays it is hard to find a business process that doesn't use software for improvement.
Various technologies come to be valued because of this.
A recent trend is to use Deep Learning for types of problems that range from self driving
cars to medical diagnosis~\cite{balas_handbook_2019}.
Deep Learning is a powerful technology based on Artificial Neural Networks where data is processed
in multiple layers to extract features and solve a given problem~\cite{shrestha_review_2019}.
One area where this is especially helpful is the field of Computer Vision.
Deep Learning for Computer Vision has only caught on in the recent years as the big computational
cost has been met by the improvement in computer hardware~\cite{ponti_everything_2017}.
Computer vision deals with extracting information from photos.
This includes tasks like recognizing faces or reading text~\cite{prince_computer_2012}.
Applying Deep Learning to extract equipment labels from photos fits right into this crease of
applying technology for making daily problems more efficient.
Combining the two fields to create value and learning about the underlying theoretical foundations
and inner workings is the motivating factor of this work.

\section{Problem description}\label{se:problem}
% TODO: add costs
% TODO: cut down this part -> put into EDA part
% TODO: define requirements better
Motivated by the wide success of Deep Learning concerning Computer Vision,
the objective of this work is to implement and train a Deep Learning model that can extract
equipment names from photos taken of name plates.

When determining whether automisation is an improvement four aspects have to be examined.
These are time, costs, quality and flexibility.
The aspects build a quadrangle that is based on the optimizing trade-off between the
factors~\cite{dumas_fundamentals_2013}.

Without software supporting the task of reading the name of the picture  and typing it into
the system, can take long seconds, whereas a trained Deep Learning model could complete the task
in a mere instant.
Therefor automisation via Deep Learning should improve the efficiency of the process when compared to
manually reading and typing the information off the image.

Training costs for a Deep Learning model are very high due to the computing intensive
backpropagation algorithm that tunes the network to the data.
But the usage cost is low.
For manual labor the opposite is the case as training a person to type in a label is done quickly
and labor costs are high in comparison to the expenses for running the model.

Both Deep Learning models and human labor are not 100\% accurate.
It is human to make mistakes and because Deep Learning is trained only trained on a specific set
of data it makes sense that not all predictions can be correct as there can always be outliers in
the data. % TODO: find source
The question is whether the model can be as accurate or even better than its human counterpart.
This is especially interesting when it is applied in the real world where it might have to do good
in subpar situations.
An example is bad image quality.

Flexibility is concerned with how well a process can adjust to changing requirements.
A set of new equipment names that have to be included can pose a problem to a Deep Learning model
because it is not trained for the new data.
A human on the other hand should not have any problems in this regard.

The main concern for the solution's efficacy is whether it is accurate enough.
Therefor this work focuses on this aspect in particular.

\section{Methodology}

The goal of this work is to implement and train a Deep Learning model to read in labels from photos.
The emerging artifact can be used to solve the problem detailed in~\ref{se:problem}.
The expository instantiation is helpful to gain more understanding the artifact as it is common
in design science.
In particular this is justificatory knowledge on the design on the Deep Learning model and
Machine Learning way of approaching problems.
This is important in order to apply it and to optimize existing research to the specific problem.

The methodology is based on action research~\cite{johannesson_introduction_2021}.
It constists of a cycle of five phases: Diagnosis, Planning, Intervention, Evaluation, Reflection.
% TODO: add how I get data
The first cycle will entail an exploratory data analysis which corresponds to the Diagnosis part.
Here it is important to recognize main characteristics of the images and to find outliers
and other potential problems~\cite{cox_translating_2017}.
The research is then extended to existing practical solutions for similar practical problems as
well as proposed architectures from academic research.
Theoretical knowledge about the models as well as practical information about results for
similar problems contribute to the discussion about which approach is the most promissing.
Combining architectures is also a viable possibility to solve the given problem.
This concludes the Planning phase and will lead to a model exaptation that evolves to be the
artifact at the center of this thesis.
The next step is implementing and training the chosen approach which.
Evaluation for of the current model follows.
Storing and analyzing results of training and cross validation as well as visualizing the training
progress is an important part of this.
In the Reflection stage it is decided whether a new cycle should be carried out.

From the second cycle on the first three phases change as there already is a model that is to be
improved.
This time the Diagnosis phase entails asking questions about the existing model: What worked?
Why did it work/not work? What needs to change?
Changes are planned and implemented accordingly.
The Evaluation and Reflection phases are not changing in the second cycle thus closing the loop.
The incremental adjustments to the model are made in order to improve the accuracy.
This includes possibly adjusting the architecture, hyperparameter tuning and preprocessing approaches
like image compression.

\section{Expected results and outlook}
The research into the theoretical foundation of Deep Learning and into possible approaches leads
to a strong understanding of the underlying technology.
This is helpful to produce a comparison of approaches that is based on theoretical as well as
practical knowledge.
The goal is to find out which approach work best for the chosen practical problem and why that is the
case.
Implementation and training of the most promissing one is yielding the artifact this work revolves
around.
The process of optimization not only improves the solution to the problem (see~\ref{se:problem}) but
is also used to learn more about the implemented approach.

Integrating the artifact into the business process is not an issue that is discussed in
this work.
Nor will model feedback and over time iteration be part.

The intendet structure of the thesis with dependencies between chapters can be found in
figure~\ref{fig:chapters}.

\begin{figure}[ht]
    \centering
    \includegraphics[width=1.0\textwidth]{img/Chapter.drawio.png}
    \caption{Chapters with subchapters and dependencies\label{fig:chapters}}
\end{figure}

\chapter{Theoretical Foundation}\label{ch:theoretical}
This chapter succinctly describes principles which build the foundation for later chapters.
Only the most relevant topics are touched upon, as the details are explained in later chapters.
The mathematics that makes the techniques possible is not explained in depths.

\section{Machine Learning}
In order to grasp \ac{DL}, a solid understanding of \ac{ML} has to be developed
first~\citep{goodfellow_deep_2016}.
This is because \ac{DL} is a subfield of \ac{ML}~\citep{chauhan_review_2018}.
The most well known definition for \ac{ML} comes from~\cite{mitchell_machine_1997}:
`A computer program is said to learn from experience $E$ with respect to some class of tasks $T$
and performance measure $P$, improves with experience $E$'.

% XXX: explain difference: model - algorithm
The task that the \ac{MLS} learns to perform, can range from approximating a function
(e.g.\ regression --- $f:\R^n \rightarrow \R$, classification ---
$f:\R^n \rightarrow \{1,\ldots,k\}$) to optaining a different representation for the data that
has beneficial properties for further processing but preserves as much information as possible
(e.g.\ PCA for compression)~\citep{goodfellow_deep_2016}.
Note that the learning itself is not the task but merely the process of improving on performing the
task~\citep{goodfellow_deep_2016}.
One of the most well known \ac{ML} algorithms is Linear Regression.
In the following the algorithm is used as an example for explaining \ac{ML} principles.
As the name implies, Linear Regression is used to predict a value $\hat{y}$ given the input vector
$\x\in\R^n$ which is made up of the features $x_i$.
The goal is to approximate the ground truth $y$.
Linear is derived from the underlying model shown in Equation~\ref{eq:linReg}:
\begin{equation}\label{eq:linReg}
    f(\x;\w,b) = \w^{T} \cdot \x + b = \sum_{i=1}^{n} w_i x_i + b = \hat{y}
\end{equation}
The scalar product of the weights $\w\in\R^n$ and \x\ is added to the bias term $b\in\R$.
Both $\w,b$ are parameters that are learned by the model in order to optimize the
approximation~\citep{goodfellow_deep_2016}.
% FIXME: here figure with linear regression

The performance of a model measures how well the task can be completed.
Depending on the task of the \ac{MLS}, different quantitative measures are used.
The metric Mean Squared Error (see Equation~\ref{eq:mse}) can be used for Linear Regression.
\begin{equation}\label{eq:mse}
    MSE =\frac{1}{m}\norm{{(\hat{\textbf{y}} - \textbf{y})}}^2
        =\frac{1}{m}\sum_{i=1}^m {((\w^T \x^{(i)} + b) - \yti)}^2
\end{equation}
Here $m$ denotes the number of examples $\X$ with the associated targets \y, used to calculate
the error~\citep{geron_hands-machine_2017,goodfellow_deep_2016}.
The goal is to minimize the generalization error which measures the expected performance on
previously unseen input~\citep{geron_hands-machine_2017}.
For this the test set is used, once the model has been trained.
The test set is a part of the available data~\citep{geron_hands-machine_2017, goodfellow_deep_2016}.
The generalization error can be divided into three components.
The bias error arises from simplifying assumptions for the model, the variance error measures the
variation in the model outcome depending on the data used for training.
Both these errors are influenced by the model's capacity which is why the relationship between them
is call the Bias/Variance tradeoff.
Lastly the irreducible error stems from not having measured all data as well as the variation
in real data and cannot be
reduced~\citep{ashmore_assuring_2021, james_introduction_2013,geron_hands-machine_2017}.

The experience part of \ac{ML} depicts the process where the algorithm is `experiencing' the training
dataset $\Xt$ and is learning important properties of the dataset.
In general there are two paradigms for training: supervised and
unsupervised~\citep{goodfellow_deep_2016}.
Linear Regression is an example for supervised learning, as the model is approximating the target
value $\yti$\ for the associated input $\xti$~\citep{alzubi_machine_2018,goodfellow_deep_2016}.
For unsupervised learning on the other hand the algorithm is not directed to predict a target
value but to learn properties about the data and to leverage them for representation tasks
like compressing or denoising the data~\citep{goodfellow_deep_2016}.
The important difference between the paradigms is that the unsupervised learning algorithm does
not use a target value, there's no ground truth
value~\citep{goodfellow_deep_2016,geron_hands-machine_2017}.
In most cases training can be described as an optimization problem, i.e.\ as minimizing a
function --- the so called objective or loss function~\citep{goodfellow_deep_2016}.
The MSE introduced earlier can be used for Linear Regression (see Equation~\ref{eq:mseOpt}).
This objective function has properties which make it suitable for models which have linear
output~\citep{goodfellow_deep_2016}.
\begin{equation}\label{eq:mseOpt}
    \min_{\w,b} MSE(\w,b)
\end{equation}
Note that for minimization the MSE is a function of $\w,b$ and not of $\x$, in terms of
predicting a value the MSE is a function of $\x$ parametrized by $\w,b$.
In Equation~\ref{eq:linReg} \w,$b$ are parameters that have to be learned in order to minimize
the generalization error~\citep{james_introduction_2013,geron_hands-machine_2017}.
For other tasks such as binary classification, the metric (e.g. $F_1$-Score) and the
objective function (binary cross entropy loss) are different~\citep{geron_hands-machine_2017,
ho_real-world-weight_2020}.
% FIXME: simple gradient descent figure / learning process graphic
For optimization the \ac{GD} algorithm is prevalent, especially in the subfield of \ac{DL}.
As the name suggests, the gradient is used to iteratively update the parameters $\w,b$ to arrive
at a minimum of the objective function (see Equation~\ref{eq:gradDescW}
and~\ref{eq:gradDescb})~\citep{geron_hands-machine_2017}.
\begin{equation}\label{eq:gradDescW}
    \w \leftarrow \w - \epsilon \cdot \nabla_{\w} MSE(\w,b) = \w-\frac{2\epsilon}{m}\Xt^T (\Xt\w+b-\y)
\end{equation}
\begin{equation}\label{eq:gradDescb}
    b \leftarrow b - \epsilon \cdot \frac{\delta}{\delta b} MSE(\w,b) = b-\frac{2\epsilon}{m}(\Xt\w+b-y)
\end{equation}
The learning rate constant $\epsilon$ can be adjusted to speed up or slow down the `steps' which
can have different effects on the convergence~\citep{goodfellow_deep_2016}.
There are more sophisticated variations of the \ac{GD} algorithm which are more suited for practical
application (e.g. RMSProp, Adam)~\citep{geron_hands-machine_2017}.
Note that the process minimizes the test error with the test set $\Xt$.
The effect on the generalization error depends on model capacity which is the space of functions
the model enables~\citep{goodfellow_deep_2016}.
Linear Regression has the capacity to fit to data with a linear relationship between features and
ground truth.
If the underlying relationship is more complicated, the model can only underfit the data (model
bias)~\citep{goodfellow_deep_2016}.
Polynomial Regression has more capacity for example.
Say the real relationship between features and ground truth now actually is linear;
the Polynomial Regression model can overfit for statistical outliers in the test set which is why
in this case the model with the lower capacity can achieve a lower generalization
error~\citep{geron_hands-machine_2017}.
Therefore, it is important to improve the bias/variance tradeoff.
% FIXME: graphic for underfitting/overfitting -> LinRegr - Under, PolRegr - Over / Fig 5.3 goodfellow

\section{Deep Learning}
approximate a function

` One of the main differences from traditional ma- chine learning (ML) methods is that DL
automatically learns how to represent data using multiple layers of abstraction [5], [6].
In traditional ML, a significant amount of work has to be spent on “feature engineering” to
build this representation manually, but this process can now be automated to a higher degree.
Having an automated and data-driven method for learning how to represent data improves both the
performance of the model and reduces requirements for manual feature engineering work
[7], [8].'~\citep{arpteg_software_2018}

\begin{enumerate}
    \item ANN / MLP % Node, Feedforward, Backpropagation / Optimization
        \begin{itemize}
            \item Architecture $\rightarrow$ Input, Hidden, Output
            \item Feedforward
            \item Optimization $\rightarrow$ Backpropagation, SGD, ADAM, \ldots
        \end{itemize}
    \item Regularization: L0,L1,L2, Dropout, Dropconnect
    \item important architectures
        \begin{itemize}
            \item CNN % layers --- convolutional, max-pooling
            \item RNN % recurrent layer
            \item transformer
            \item Specific foundation architectures for relevant approaches
        \end{itemize}
    \item transfer learning: reuse parameters from pretrained models\\
\end{enumerate}

\section{Optical Character Recognition}

\begin{enumerate}
    \item def
    \item little history
    \item need for \ac{STS}
    \item evaluation metric and matching prediction to ground truth
    \item common data sets
\end{enumerate}

\chapter{Exploratory Data Analysis}
%  Python Software für Layout-Detection -> suchen

% TODO: add costs
% TODO: speed / accuracy tradeoff
% TODO: cut down this part -> put into EDA part
When determining whether automisation is an improvement four aspects have to be examined.
These are time, costs, quality and flexibility.
The aspects build a quadrangle that is based on the optimizing trade-off between the
factors~\cite{dumas_fundamentals_2013}.

Without software supporting the task of reading the name of the picture  and typing it into
the system, can take long seconds, whereas a trained Deep Learning model could complete the task
in a mere instant.
Therefor automisation via Deep Learning should improve the efficiency of the process when compared to
manually reading and typing the information off the image.

Training costs for a Deep Learning model are very high due to the computing intensive
backpropagation algorithm that tunes the network to the data.
But the usage cost is low.
For manual labor the opposite is the case as training a person to type in a label is done quickly
and labor costs are high in comparison to the expenses for running the model.

Both Deep Learning models and human labor are not 100\% accurate.
It is human to make mistakes and because Deep Learning is trained only trained on a specific set
of data it makes sense that not all predictions can be correct as there can always be outliers in
the data. % TODO: find source
The question is whether the model can be as accurate or even better than its human counterpart.
This is especially interesting when it is applied in the real world where it might have to do good
in subpar situations.
An example is bad image quality.

Flexibility is concerned with how well a process can adjust to changing requirements.
A set of new equipment names that have to be included can pose a problem to a Deep Learning model
because it is not trained for the new data.
A human on the other hand should not have any problems in this regard.

The main concern for the solution's efficacy is whether it is accurate enough.
Therefor this work focuses on this aspect in particular.

\chapter{System Design}

Search for specific information
% Vergleich muss gut genug sein

\section{Approach comparison}
% include Pipeline differences

% why new model? important what it is trained on, most for documents, recognizing random alphanumeric
% sequences

\subsection{Approach Research}
\subsubsection*{GitHub implementation}
Two models that can be used in conjunction

\textbf{detection}~\cite{beom_text_2021}\\
uses RetinaNet structure~\cite{lin_focal_2018}
applies techniques from textboxes++~\cite{liao_textboxes_2018}

\textbf{character recognition}~\cite{beom_crnn_2021}\\
needs cropped text area as input\\
uses CRNN~\cite{shi_end--end_2015} $\rightarrow$ end-to-end learning, LSTM fir arbitrary length of
input and output, no need to apply detection and cropping to each single character

\subsubsection*{Tesseract}
Open Source OCR engine~\cite{smith_overview_2007}
\begin{itemize}
    \item uses Deep Learning (found c++ code for layers in repo)
    \item Processing in step-by-step pipeline, some unusual stages\\
        1. Line and Word finding\\
        1.1. Line finding\\
        1.2. Baseline Fitting\\
        1.3. Fixed Pitch Detection and Chopping\\
        1.4. Proportional Word Finding\\
        2. Word Recognition\\
        2.1 Chopping Joined Characters\\
        2.2 Accociating Broken Characters\\
        3. Static Character Classifier\\
        3.1 Features\\
        3.2 Classification\\
        3.3 Training Data\\
        4. Linguistic Analysis\\
        5. Adaptive Classifier
\end{itemize}

\subsubsection*{EAST}
An Efficient and Accurate Scene Text Detector



\subsection{Comparison}

\section{Approach selection}

\chapter{Implementation}
\section{Software and Tools}
\section{Preprocessing}
\section{Prototype}
% DataSet suche online
\section{Optimizations}
% pretrained weights
% bias variance tradeoff

\chapter{Discussion}
\section{Results}
\section{Method reflection}

\chapter{Conclusion}

%Integrating the artifact into the business process is not an issue that is discussed in
%this work.
%Nor will model feedback and over time iteration be part.


\appendix

\chapter{References}

% Abbildungsverzeichnis (kann auch nach dem Inhaltsverzeichnis kommen)
\listoffigures

% delete group thing to have tables on new page
\begingroup
\let\clearpage\relax
% Tabellenverzeichnis (kann auch nach dem Inhaltsverzeichnis kommen)
\listoftables
\endgroup

% Literaturverzeichnis
\bibliographystyle{dbstmpl}    % verwendet dbstmpl.bst

% alternative, vorinstallierte Stile sind z.B. plain oder abbrv
\bibliography{dbstmpl}         % verwendet dbstmpl.bib

\chapter{Code}

code here

\end{document}
